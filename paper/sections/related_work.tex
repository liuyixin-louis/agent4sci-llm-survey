\section{Related Work}

\subsection{AI-Assisted Literature Review}

Recent years have witnessed growing interest in using AI systems to assist with literature review tasks. Several approaches have emerged, each addressing different aspects of the literature surveying process.

\subsubsection{Retrieval-Augmented Generation (RAG) Systems}

RAG-based approaches combine information retrieval with large language models to generate literature summaries. Works such as \cite{lewis2019retrieval} and \cite{guu2020retrieval} demonstrate how retrieval mechanisms can enhance LLM outputs with factual information. However, these systems typically focus on individual document summarization rather than comprehensive literature surveys across multiple sources.

\subsubsection{AI-Powered Systematic Reviews}

Several studies have explored automated systematic review processes. \cite{marshall2017automation} developed tools for automating systematic review workflows, while \cite{van2019automation} focused on automating study selection and data extraction. These approaches, however, maintain human oversight and do not fully automate the synthesis and writing phases.

\subsubsection{LLM-Based Literature Analysis}

Recent work has explored using LLMs for literature analysis tasks. \cite{wang2023large} demonstrated LLM capabilities in understanding scientific literature, while \cite{liu2023survey} explored using LLMs for survey generation. However, these approaches lack the systematic, multi-agent coordination that our pipeline provides.

\subsection{Agent Frameworks and Multi-Agent Systems}

Our work builds upon established research in agent frameworks and multi-agent coordination systems.

\subsubsection{Autonomous Agent Frameworks}

Frameworks like AutoGPT \cite{significant2023autogpt} and LangChain \cite{langchain2022} have popularized the concept of autonomous AI agents. These systems demonstrate the potential for LLM-based agents to perform complex, multi-step tasks. However, they often lack the specialized domain knowledge and systematic approach required for scientific literature analysis.

\subsubsection{Multi-Agent Coordination}

Research in multi-agent systems \cite{wooldridge2009introduction} provides theoretical foundations for coordinating multiple specialized agents. Our pipeline applies these principles to create a coordinated system where each agent has a specific role in the literature surveying process.

\subsubsection{Scientific Workflow Automation}

Several projects have explored automating scientific workflows. \cite{gil2011workflow} discusses workflow automation in scientific computing, while \cite{deelman2015pegasus} presents frameworks for scientific workflow management. Our work extends these concepts to literature surveying, introducing AI agents as workflow components.

\subsection{Literature Survey Methodologies}

\subsubsection{Traditional Survey Approaches}

Conventional literature surveys follow established methodologies \cite{kitchenham2004procedures}, typically involving manual search, screening, and synthesis. While effective, these approaches struggle with the volume and velocity of modern research output, particularly in fast-moving fields like LLM research.

\subsubsection{Systematic Review Automation}

Efforts to automate systematic reviews have focused on specific stages of the process. \cite{wallace2016identifying} automated study identification, while \cite{thomas2011automating} focused on data extraction. Our approach differs by providing end-to-end automation with AI agents handling all stages.

\subsubsection{Meta-Science and Research Synthesis}

The field of meta-science \cite{ioannidis2018meta} examines how research is conducted and synthesized. Our work contributes to this area by exploring how AI systems can participate in research synthesis, potentially accelerating the pace of scientific discovery.

\subsection{Evaluation of AI-Generated Content}

\subsubsection{Quality Assessment Metrics}

Evaluating AI-generated content requires appropriate metrics. \cite{zhang2023evaluating} discusses evaluation approaches for AI-generated text, while \cite{liu2023survey} proposes metrics specific to survey quality. We build upon these frameworks to assess our pipeline's output.

\subsubsection{Comparison with Human-Authored Content}

Understanding how AI-generated surveys compare to human-authored ones is crucial. \cite{wang2023large} provides frameworks for such comparisons, while \cite{liu2023survey} demonstrates evaluation methodologies. Our evaluation extends these approaches to comprehensive literature surveys.

\subsection{Gaps in Current Research}

Despite significant progress in AI-assisted literature review, several gaps remain:

\begin{itemize}
    \item \textbf{End-to-End Automation}: Current approaches focus on individual stages rather than complete automation
    \item \textbf{Multi-Source Integration}: Limited work exists on systematically combining multiple literature sources
    \item \textbf{AI Authorship Transparency}: Few studies address the ethical and practical implications of AI-authored scientific content
    \item \textbf{Scalability}: Existing solutions struggle with the exponential growth in research output
\end{itemize}

Our work addresses these gaps by providing a comprehensive, agentic pipeline that can autonomously conduct literature surveys while maintaining transparency about AI contributions.
