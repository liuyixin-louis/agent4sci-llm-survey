\section{Introduction}

The field of Large Language Models (LLMs) has experienced unprecedented growth, with thousands of research papers published annually across multiple disciplines including natural language processing, computer vision, robotics, and interdisciplinary applications. This rapid expansion presents a significant challenge for researchers attempting to stay current with the state-of-the-art: traditional manual literature surveys become outdated almost as quickly as they are completed.

\subsection{The Challenge of Manual Literature Surveys}

Current approaches to literature surveying in LLM research face several critical limitations:

\begin{itemize}
    \item \textbf{Volume Overwhelm}: The exponential growth in LLM publications makes comprehensive coverage nearly impossible for individual researchers
    \item \textbf{Temporal Lag}: Human-authored surveys typically take 6-12 months to complete, during which the field may have evolved significantly
    \item \textbf{Subjectivity}: Human reviewers bring inherent biases and may miss important papers outside their immediate research focus
    \item \textbf{Reproducibility}: Manual survey processes are difficult to replicate and validate
\end{itemize}

\subsection{The Promise of AI-Generated Surveys}

Recent advances in LLM capabilities suggest a potential solution: using AI systems to autonomously conduct literature surveys. This approach offers several advantages:

\begin{itemize}
    \item \textbf{Scalability}: AI systems can process thousands of papers simultaneously
    \item \textbf{Timeliness}: Automated pipelines can generate surveys in days rather than months
    \item \textbf{Systematic Coverage}: AI can apply consistent criteria across large corpora
    \item \textbf{Reproducibility}: Automated processes can be exactly replicated
\end{itemize}

\subsection{Research Questions and Contributions}

This paper addresses the following research questions:

\begin{enumerate}
    \item Can LLM-based agents autonomously conduct high-quality literature surveys?
    \item How do AI-generated surveys compare to human-authored surveys in terms of coverage, accuracy, and insight?
    \item What are the limitations and ethical considerations of AI-authored scientific literature?
\end{enumerate}

Our primary contributions include:

\begin{itemize}
    \item \textbf{Agentic Pipeline Design}: A novel multi-agent system for autonomous literature surveying
    \item \textbf{Comprehensive Evaluation}: Systematic comparison of AI vs. human survey quality
    \item \textbf{Transparency Framework}: Clear disclosure of AI contributions and limitations
    \item \textbf{Open Implementation}: Publicly available code for replication and extension
\end{itemize}

\subsection{Paper Organization}

The remainder of this paper is organized as follows: Section II reviews related work in AI-assisted literature review and agent frameworks. Section III describes our agentic pipeline methodology. Section IV presents results from surveying LLM research (2020-2025). Section V discusses implications, limitations, and ethical considerations. Section VI concludes with future directions.
