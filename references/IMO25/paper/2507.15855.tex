\title{Gemini 2.5 Pro Capable of Winning Gold at IMO 2025\footnote{Code available at: \url{https://github.com/lyang36/IMO25}}}
\author{Yichen Huang (黄溢辰)\thanks{\href{mailto:huangtbcmh@gmail.com}{huangtbcmh@gmail.com}} \and Lin F. Yang (杨林)\thanks{\texttt{linyang@ee.ucla.edu}, Department of Electrical and Computer Engineering, and Department of Computer Science, UCLA}}
\date{\today}

\begin{document}
\begin{CJK}{UTF8}{gbsn}
\maketitle
\end{CJK}
\begin{abstract}
The International Mathematical Olympiad (IMO) poses uniquely challenging problems requiring deep insight, creativity, and formal reasoning. While Large Language Models (LLMs) perform well on mathematical benchmarks like AIME, they struggle with Olympiad-level tasks. We use Google’s Gemini 2.5 Pro on the newly released IMO 2025 problems, avoiding data contamination. Using a self-verification pipeline with careful prompt design, 5 (out of 6) problems are solved correctly. This result underscores the importance of developing optimal strategies to harness the full potential of powerful LLMs for complex reasoning tasks.
\end{abstract}

\section{Introduction}
The International Mathematical Olympiad (IMO)~\cite{imo_official} is an esteemed annual competition that convenes the world's most talented pre-university mathematicians. Established in Romania in 1959 with just seven participating countries, it has since expanded to include over 100 nations, each represented by a team of up to six contestants. Held annually, with the sole exception of 1980, the IMO challenges participants with exceptionally difficult problems in fields like algebra, geometry, number theory, and combinatorics. Contestants are given two 4.5-hour sessions over two days to solve three problems per session, each graded out of seven points.
Unlike typical mathematical exercises, IMO problems demand profound insight, originality, and the ability to synthesize diverse mathematical concepts. This emphasis on creative, proof-based reasoning makes the IMO a hallmark of mathematical excellence and a vital platform for identifying future leaders in the field. 

Consequently, the IMO has also become a grand challenge and a formidable benchmark for evaluating the advanced reasoning capabilities of Artificial Intelligence, particularly Large Language Models (LLMs), providing a rigorous test of their ability to perform complex, multi-step logical deduction rather than rote calculation~\cite{he2024olympiadbench, trinh2024solving, matharena2025}. Traditional benchmarks like GSM8K and MATH focus on grade-school and high-school level problems, respectively, where LLMs have achieved high performance through pattern recognition and retrieval from training data~\cite{cobbe2021gsm8k,hendrycks2021math}. However, IMO problems surpass these in complexity, requiring multi-step reasoning, abstraction, and innovation akin to human expert-level cognition, thereby exposing limitations in LLMs’ generalization and vulnerability to hallucinations or superficial heuristics~\cite{frontiermath2024}. This positions the IMO as an ideal probe for assessing whether LLMs can truly ``reason’’ rather than merely replicate memorized solutions, addressing concerns about their reliability in high-stakes domains like scientific discovery and formal verification~\cite{li2025advocate}.

The pursuit of automated mathematical reasoning has seen remarkable progress with the advent of LLMs~\cite{ahn2024large, wang2025survey}. Early successes on foundational benchmarks have rapidly escalated to tackling complex, competition-level mathematics. This progress has been significantly propelled by innovations such as Chain-of-Thought (CoT) prompting, which enables models to generate intermediate reasoning steps, thereby improving performance on tasks requiring complex logic and calculation~\cite{wei2022chain}. Nevertheless, even state-of-the-art models have demonstrated significant limitations when confronted with Olympiad-level problems. For example, recent evaluations on problems from the USA Mathematical Olympiad (USAMO) 2025 and IMO 2025 showed that top-tier public models still struggle to produce sound, rigorous proofs and fail to achieve scores comparable to human medalists, often succumbing to logical fallacies and a lack of creative insight~\cite{petrov2025proof, matharena2025}. This highlights a critical gap between generating numerically correct answers and constructing logically sound arguments~\cite{mahdavi2025brains}.

In this paper, we construct a self-verification pipeline with careful prompt design and implemented using the Gemini 2.5 Pro model, a strong base model released by Google~\cite{gemini2025blog}. We solved 5 out of the 6 problems of IMO 2025. A persistent and critical challenge in the evaluation of LLMs is the issue of data contamination, where test data from public benchmarks is inadvertently included in the vast pre-training corpora, leading to inflated and unreliable performance metrics~\cite{zhou2024dont}. To ensure a rigorous and uncontaminated assessment of the model’s genuine problem-solving capabilities, this work exclusively utilizes the problems from the most recent IMO 2025 competition. As these problems were released only days before our evaluation, they serve as a pristine testbed, mitigating the risk of data leakage and providing a robust measure of the model’s ability to generalize and reason on genuinely unseen challenges. Our approach shows that strong existing models are already capable of solving difficult math reasoning problems, but directly using them can result in poor results as shown in~\cite{matharena2025}. Our results demonstrate a significant advance in automated mathematical reasoning.

Concurrent with our work, other teams also reported high-level performance on the IMO 2025 problems. These include OpenAI~\cite{wei2025openai}, Google DeepMind~\cite{Deepmind}, and ByteDance~\cite{bytedance2025seed}.

\section{Methods}

\subsection{Pipeline}

At a high level, our pipeline proceeds as follows (illustrated in Figure \ref{fig:flow-diagram}):

\begin{itemize}
    \item Step 1: Initial solution generation with the prompt in Section~\ref{prompt:step1};
    \item Step 2: Self-improvement;
    \item Step 3: Verifying the solution with the prompt in Section~\ref{prompt:verification} and generating a bug report; go to Step 4 or Step 6 (see below for explanations);
    \item Step 4: Review of the bug report (optional);
    \item Step 5: Correcting or improving the solution based on the bug report; go to Step 3;
    \item Step 6: Accept or Reject.
\end{itemize}

\begin{figure}[ht]
\centering
\begin{tikzpicture}[
  scale=0.8,
  transform shape,
  every node/.style={font=\small, align=center},
  process/.style={rectangle, draw, rounded corners, minimum width=3.5cm, minimum height=1cm, text width=3.5cm},
  decision/.style={diamond, draw, aspect=2, inner sep=1pt, minimum height=1.5cm, text width=4cm},
  ->, >=Stealth
  ]

\node[process] (step1) at (0,0) {Step 1:\\ Initial solution generation};
\node[process] (step2) at (6,0) {Step 2:\\ Self-improvement};
\node[decision] (step3) at (15, 0) {Step 3:\\ Verification\\ (Go to Step 4 or 6)};
\node[process] (step4) at (2, -2) {Step 4:\\ Bug report review (optional)};
\node[process] (step5) at (2, -4) {Step 5:\\ Correction};
\node[process] (step6) at (16, -4) {Step 6:\\ Accept};
\node[process] (step7) at (10,-4) {Step 6$'$:\\ Reject};

\draw[->] (step1) -- (step2);
\draw[->] (step2) -- (step3);
\draw[->] (step3) -- (step4) node[midway, left] {\textcolor{blue}{failed}};
\draw[->] (step3) -- (step6) node[midway, right] {\textcolor{blue}{consecutively passes 5 times}};
\draw[->] (step3) -- (step7) node[midway] {\textcolor{blue}{w/ major issue for 10 steps}};
\draw[->] (step4) -- (step5);
\draw[->] (step5) -- (step3);

\end{tikzpicture}
\caption{Flow diagram of our pipeline. See the main text for detailed explanations of each step.}
\label{fig:flow-diagram}
\end{figure}

We run the procedure some number of times (in parallel or in serial, independently) to obtain a correct solution. We hope the model either outputs the correct solution, or claims that it fails to identify the solution.

\subsection{Solver}
The solver prompt in Section~\ref{prompt:step1} for Step 1 is designed to emphasize rigor rather than focus on finding the final answer and thus matches the theme of IMO. We have randomly selected some outputs of this step and found that the overall quality of the solutions are pretty low. This is consistent with very recent findings of Ref.~\cite{matharena2025}.

In Step 2, the model is prompted to review and try to improve its work. While Gemini 2.5 Pro is good at mathematics, as a general-purpose LLM, it is not tailored to solving especially challenging mathematical problems. One significant constraint is the thinking budget. Note that thinking is quite token consuming: Even a trivial fact might take a few thousand tokens for the model to prove. The maximum number of thinking tokens of Gemini 2.5 Pro is 32768, which is not enough for solving a typical IMO problem. We observe that in Step 1, the model almost always uses up its thinking budget. Thus, the model does not even have the capacity to fully solve the problem. This is why we choose to break down the problem solving process into steps. Step 2 effectively injects another budget of 32768 thinking tokens to allow the model review and continue its work. We keep monitoring the entire process and do observe that the outputs have been noticeably improved during Step 2.

Next we will use the verifier to make iterative improvement and decide whether to accept an improved solution.

\subsection{Verifier}

The verifier plays an important role in our pipeline. Its functionality is to carefully review a solution step by step and find out issues (if any). We emphasize mathematical rigor and classify issues into critical errors and justification gaps. Critical errors are something that is demonstratively false or with clear logical fallacies, while justification gaps can be major or minor. A major justification gap that cannot be repaired would crash an entire proof, while minor justification gaps may not even be well defined: A minor gap could sometimes be viewed as concise argument.

In Step 3, we use the verifier to generate a bug report for each solution outputted in Step 2. The bug report contains a list of issues classified as critical errors or justification gaps. For each issue, an explanation is required. The bug report will serve as useful information for the model to improve the solution, either fixing errors or filling gaps. Step 4 (optional) is to carefully review each issue in the bug report. If the verifier makes a mistake and reports an issue which is not really an issue, the issue would be deleted from the bug report. Thus, Step 4 increases the reliability of the bug report. In Step 5, the model tries to improve the solution based on the bug report. We iterate Steps 3-5 a sufficient number of times until we decide to accept or decline a solution. We accept a solution if it robustly passes the verification process and decline a solution if there are always critical errors or major justification gaps during the iterations.

We observe that the verifier is quite reliable but can make mistakes. Since our major goal is not to benchmark the verifier, we do not have quantitative results on its effectiveness. However, we have used this verifier for quite a while (starting from well before IMO 2025). We have been keeping an eye on its performance and below is our qualitative observation:
\begin{itemize}
    \item Critical errors are seldom missed by the verifier. This is consistent with the observations in Refs.~\cite{SLJ15,guo2025temporal}. In the unlikely event such errors are not caught, simply running the verifier a few more times would very likely catch it. This is good because we do not wish to miss critical errors.
    \item If the verifier reports a critical error, it may not always be critical, but it almost always needs some revision.
    \item The verifier may report some justification gaps which are only slightly beyond trivial statements and thus are not really gaps for mathematicians.
\end{itemize}
Indeed, our system is quite robust to errors made by the verifier. We iteratively use the verifier a sufficiently number of times. If it misses an error in one iteration, it still has some probability to catch it in the next iteration. Also, if it claims an error which is actually not an error, such a false negative may not go through the bug report review step (Step 4). Furthermore, we instruct the model (who generates the solution) to review each item in the bug report. If the model does not agree with a particular item, it is encouraged to revise its solution to minimize misunderstanding. This is analogues to the peer review process. If a referee makes a wrong judgment, the authors are encouraged to revise the paper. Ultimately, the presentation is improved.

At the time we plan to accept a solution, we do not wish the verifier misses any issue; we run the verifier five times and accept a solution only if it passes every time.

\section{Experiment Setup}
We choose low temperature: $0.1$. High temperature may lead to more random errors, which may be harmful. We use the maximum thinking budget ($32768$ reasoning tokens) of Gemini 2.5 Pro. We do not use web search (of course), code, or any other tools. We share the most important prompts below.

\subsection{Step 1 Prompt}
\label{prompt:step1}
\begin{lstlisting}
    ### Core Instructions ###

*   **Rigor is Paramount:** Your primary goal is to produce a complete and rigorously justified solution. Every step in your solution must be logically sound and clearly explained. A correct final answer derived from flawed or incomplete reasoning is considered a failure.
*   **Honesty About Completeness:** If you cannot find a complete solution, you must **not** guess or create a solution that appears correct but contains hidden flaws or justification gaps. Instead, you should present only significant partial results that you can rigorously prove. A partial result is considered significant if it represents a substantial advancement toward a full solution. Examples include:
    *   Proving a key lemma.
    *   Fully resolving one or more cases within a logically sound case-based proof.
    *   Establishing a critical property of the mathematical objects in the problem.
    *   For an optimization problem, proving an upper or lower bound without proving that this bound is achievable.
*   **Use TeX for All Mathematics:** All mathematical variables, expressions, and relations must be enclosed in TeX delimiters (e.g., `Let $n$ be an integer.`).

### Output Format ###

Your response MUST be structured into the following sections, in this exact order.

**1. Summary**

Provide a concise overview of your findings. This section must contain two parts:

*   **a. Verdict:** State clearly whether you have found a complete solution or a partial solution.
    *   **For a complete solution:** State the final answer, e.g., "I have successfully solved the problem. The final answer is..."
    *   **For a partial solution:** State the main rigorous conclusion(s) you were able to prove, e.g., "I have not found a complete solution, but I have rigorously proven that..."
*   **b. Method Sketch:** Present a high-level, conceptual outline of your solution. This sketch should allow an expert to understand the logical flow of your argument without reading the full detail. It should include:
    *   A narrative of your overall strategy.
    *   The full and precise mathematical statements of any key lemmas or major intermediate results.
    *   If applicable, describe any key constructions or case splits that form the backbone of your argument.

**2. Detailed Solution**

Present the full, step-by-step mathematical proof. Each step must be logically justified and clearly explained. The level of detail should be sufficient for an expert to verify the correctness of your reasoning without needing to fill in any gaps. This section must contain ONLY the complete, rigorous proof, free of any internal commentary, alternative approaches, or failed attempts.

### Self-Correction Instruction ###

Before finalizing your output, carefully review your "Method Sketch" and "Detailed Solution" to ensure they are clean, rigorous, and strictly adhere to all instructions provided above. Verify that every statement contributes directly to the final, coherent mathematical argument.
\end{lstlisting}

\subsection{Verification Prompt}
\label{prompt:verification}
\begin{lstlisting}
    You are an expert mathematician and a meticulous grader for an International Mathematical Olympiad (IMO) level exam. Your primary task is to rigorously verify the provided mathematical solution. A solution is to be judged correct **only if every step is rigorously justified.** A solution that arrives at a correct final answer through flawed reasoning, educated guesses, or with gaps in its arguments must be flagged as incorrect or incomplete.

### Instructions ###

**1. Core Instructions**
*   Your sole task is to find and report all issues in the provided solution. You must act as a **verifier**, NOT a solver. **Do NOT attempt to correct the errors or fill the gaps you find.**
*   You must perform a **step-by-step** check of the entire solution. This analysis will be presented in a **Detailed Verification Log**, where you justify your assessment of each step: for correct steps, a brief justification suffices; for steps with errors or gaps, you must provide a detailed explanation.

**2. How to Handle Issues in the Solution**
When you identify an issue in a step, you MUST first classify it into one of the following two categories and then follow the specified procedure.

*   **a. Critical Error:**
    This is any error that breaks the logical chain of the proof. This includes both **logical fallacies** (e.g., claiming that `A>B, C>D` implies `A-C>B-D`) and **factual errors** (e.g., a calculation error like `2+3=6`).
    *   **Procedure:**
        *   Explain the specific error and state that it **invalidates the current line of reasoning**.
        *   Do NOT check any further steps that rely on this error.
        *   You MUST, however, scan the rest of the solution to identify and verify any fully independent parts. For example, if a proof is split into multiple cases, an error in one case does not prevent you from checking the other cases.

*   **b. Justification Gap:**
    This is for steps where the conclusion may be correct, but the provided argument is incomplete, hand-wavy, or lacks sufficient rigor.
    *   **Procedure:**
        *   Explain the gap in the justification.
        *   State that you will **assume the step's conclusion is true** for the sake of argument.
        *   Then, proceed to verify all subsequent steps to check if the remainder of the argument is sound.

**3. Output Format**
Your response MUST be structured into two main sections: a **Summary** followed by the **Detailed Verification Log**.

*   **a. Summary**
    This section MUST be at the very beginning of your response. It must contain two components:
    *   **Final Verdict**: A single, clear sentence declaring the overall validity of the solution. For example: "The solution is correct," "The solution contains a Critical Error and is therefore invalid," or "The solution's approach is viable but contains several Justification Gaps."
    *   **List of Findings**: A bulleted list that summarizes **every** issue you discovered. For each finding, you must provide:
        *   **Location:** A direct quote of the key phrase or equation where the issue occurs.
        *   **Issue:** A brief description of the problem and its classification (**Critical Error** or **Justification Gap**).

*   **b. Detailed Verification Log**
    Following the summary, provide the full, step-by-step verification log as defined in the Core Instructions. When you refer to a specific part of the solution, **quote the relevant text** to make your reference clear before providing your detailed analysis of that part.

**Example of the Required Summary Format**
*This is a generic example to illustrate the required format. Your findings must be based on the actual solution provided below.*

**Final Verdict:** The solution is **invalid** because it contains a Critical Error.

**List of Findings:**
*   **Location:** "By interchanging the limit and the integral, we get..."
    *   **Issue:** Justification Gap - The solution interchanges a limit and an integral without providing justification, such as proving uniform convergence.
*   **Location:** "From $A > B$ and $C > D$, it follows that $A-C > B-D$"
    *   **Issue:** Critical Error - This step is a logical fallacy. Subtracting inequalities in this manner is not a valid mathematical operation.

======================================================================
### Problem ###

[Paste the TeX for the problem statement here]

======================================================================
### Solution ###

[Paste the TeX for the solution to be verified here]

======================================================================
### Verification Task Reminder ###

Your task is to act as an IMO grader. Now, generate the **summary** and the **step-by-step verification log** for the solution above. In your log, justify each correct step and explain in detail any errors or justification gaps you find, as specified in the instructions above.
\end{lstlisting}

\section{Results and Discussion}

Our pipeline successfully generated complete and mathematically rigorous solutions for 5 out of the 6 problems from the IMO 2025. The full, verbatim proofs, which constitute the primary evidence for this claim, are provided in Appendix \ref{app}.

The solutions for Problems 3, 4, and 5 were generated without any problem-specific hints. For Problems 1 and 2, we present two distinct solutions each: one generated without a hint, and one generated with a hint. The inclusion of both solution types for Problems 1 and 2 allows for a direct analysis of the trade-off between computational sampling and explicit strategic guidance. For Problem 1, the hint was to use mathematical induction; for Problem 2, it was to use analytic geometry. In arXiv versions 1 and 2 of this paper, we presented only the hint-based solutions and hypothesized that these general hints do not provide problem-specific insight but rather serve to reduce the computational search space. We argued that a comprehensive, multi-agent system would naturally dedicate resources to exploring such standard methods. The hint-free solutions, newly presented in this version, provide strong empirical validation for this hypothesis. Our success in finding these solutions, albeit with increased sampling, confirms that the hints primarily improve efficiency rather than enabling a fundamentally new capability.

Despite the high success rate, the pipeline failed to solve Problem 6. The model's output for this problem, provided in Appendix \ref{a:6}, correctly identifies the trivial upper bound of $4048$. However, its attempt to prove a matching lower bound is built on a flawed premise. The core error is the assertion that any tile must lie entirely in one of two disjoint regions: the cells to the left of the uncovered squares ($C_L$) or the cells to the right ($C_R$). This is incorrect, as a single tile can span columns that are to the left of one row's uncovered square and to the right of another's. This invalidates the subsequent proof.

The high success rate on these diverse problems validates the central thesis of this work: that a structured, iterative pipeline is essential for converting the latent capabilities of large models into rigorous mathematical proofs. This approach directly addresses the challenge, highlighted in recent findings by \cite{matharena2025}, that raw model capability alone is often insufficient for such tasks. Our results demonstrate that the iterative refinement process systematically overcomes the limitations of single-pass generation, such as finite reasoning budgets and the critical errors and justification gaps that often appear in initial drafts. The verifier-guided loop, in particular, proved essential for eliciting rigorous, trustworthy, and complete mathematical arguments from the model.

\section{Outlook}

All results reported in this work are obtained by using Gemini 2.5 Pro only. We expect that using a diverse set of leading models (e.g, Grok 4, OpenAI-o series) would yield stronger mathematical capability.

In our pipeline, we sample solutions and then try to individually improve each solution. It is expected that a multi-agent system (analogues to Grok 4 heavy) where the strengths of different solutions can be combined would lead to stronger mathematical capability.

\bibliographystyle{abbrv}
\bibliography{ref}

\appendix

\section{Verbatim Model Outputs for IMO 2025 Problems} \label{app}

This appendix contains the verbatim solutions produced by the model. To preserve authenticity, they are rendered in their original format, which is a mixture of Markdown and TeX.

\subsection{Problem 1}

\begin{problem}
A line in the plane is called \textit{sunny} if it is not parallel to any of the $x$-axis, the $y$-axis, and the line $x+y=0$.

Let $n \ge 3$ be a given integer. Determine all nonnegative integers $k$ such that there exist $n$ distinct lines in the plane satisfying both the following:
\begin{itemize}
    \item For all positive integers $a$ and $b$ with $a + b \le n + 1$, the point $(a,b)$ is on at least one of the lines; and
    \item Exactly $k$ of the lines are sunny.
\end{itemize}
\end{problem}

\subsubsection{Solution without hint}

Let $P_n = \{ (a,b) \in \mathbb{Z}^2 \mid a > 0, b > 0, a+b \le n+1 \}$ be the set of points to be covered. A line is called *sunny* if it is not parallel to the $x$-axis (slope 0), the $y$-axis (undefined slope), or the line $x+y=0$ (slope -1). Let $k$ be the number of sunny lines in a set of $n$ distinct lines covering $P_n$. Let $K_n$ be the set of all possible values of $k$ for a given integer $n \ge 3$.

**Part 1: The case $n=3$**

For $n=3$, the set of points is $P_3 = \{(1,1), (1,2), (2,1), (1,3), (2,2), (3,1)\}$. We show that $K_3 = \{0,1,3\}$.

*   **$k=0$ is possible:** The set of non-sunny lines $\mathcal{L} = \{x=1, x=2, x=3\}$ covers all points in $P_3$. For any $(a,b) \in P_3$, $a \in \{1,2,3\}$, so $(a,b)$ lies on the line $x=a$. Thus, $0 \in K_3$.
*   **$k=1$ is possible:** Consider $\mathcal{L} = \{x=1, y=1, y=x\}$. The lines $x=1$ and $y=1$ are not sunny. The line $y=x$ (slope 1) is sunny. This set of lines covers all points in $P_3$. Thus, $1 \in K_3$.
*   **$k=3$ is possible:** Consider $\mathcal{L} = \{y=x, x+2y=5, 2x+y=5\}$. The slopes are $1, -1/2, -2$. All three lines are sunny. The line $y=x$ covers $(1,1)$ and $(2,2)$. The line $x+2y=5$ covers $(1,2)$ and $(3,1)$. The line $2x+y=5$ covers $(1,3)$ and $(2,1)$. Together, these three lines cover all six points of $P_3$. Thus, $3 \in K_3$.

*   **$k=2$ is impossible:** We prove this by contradiction. Assume there exists a set of 3 lines with 2 sunny lines ($L_1, L_2$) and 1 non-sunny line ($L_3$) that covers $P_3$.

    **Lemma:** A sunny line contains at most two points of $P_3$.
    **Proof:** The sets of three or more collinear points in $P_3$ are $\{(1,1), (1,2), (1,3)\}$ (on line $x=1$), $\{(1,1), (2,1), (3,1)\}$ (on line $y=1$), and $\{(1,3), (2,2), (3,1)\}$ (on line $x+y=4$). All three of these lines are non-sunny. Any other line can intersect $P_3$ in at most two points. Thus, a sunny line contains at most two points of $P_3$.

    Now we proceed with a case analysis on the non-sunny line $L_3$.
    1.  **$L_3$ covers 3 points of $P_3$**: $L_3$ must be one of $x=1, y=1,$ or $x+y=4$.
        *   If $L_3=x=1$, it covers $\{(1,1),(1,2),(1,3)\}$. The remaining points are $R_1 = \{(2,1),(2,2),(3,1)\}$. These three points are not collinear. To cover them with two lines $L_1, L_2$, one line must cover two points. The lines defined by pairs of points in $R_1$ are $x=2$, $y=1$, and $x+y=4$. All are non-sunny, contradicting that $L_1, L_2$ are sunny.
        *   If $L_3=y=1$, it covers $\{(1,1),(2,1),(3,1)\}$. The remaining points are $R_2 = \{(1,2),(1,3),(2,2)\}$. The lines defined by pairs of points in $R_2$ are $x=1$, $y=2$, and $x+y=4$. All are non-sunny. Contradiction.
        *   If $L_3=x+y=4$, it covers $\{(1,3),(2,2),(3,1)\}$. The remaining points are $R_3 = \{(1,1),(1,2),(2,1)\}$. The lines defined by pairs of points in $R_3$ are $x=1$, $y=1$, and $x+y=3$. All are non-sunny. Contradiction.
    2.  **$L_3$ covers 2 points of $P_3$**: The remaining 4 points must be covered by two sunny lines $L_1, L_2$. By the lemma, each must cover at most 2 points. Thus, each must cover exactly 2 points. The set of 4 remaining points must be partitioned into two pairs, each defining a sunny line. The non-sunny lines covering exactly 2 points of $P_3$ are $x=2, y=2, x+y=3$.
        *   If $L_3=x=2$, it covers $\{(2,1),(2,2)\}$. The remaining points are $R_4 = \{(1,1),(1,2),(1,3),(3,1)\}$. The partitions of $R_4$ into two pairs are:
            - $\{(1,1),(1,2)\}$ (on non-sunny $x=1$) and $\{(1,3),(3,1)\}$ (on non-sunny $x+y=4$).
            - $\{(1,1),(1,3)\}$ (on non-sunny $x=1$) and $\{(1,2),(3,1)\}$ (on sunny $x+2y=5$).
            - $\{(1,1),(3,1)\}$ (on non-sunny $y=1$) and $\{(1,2),(1,3)\}$ (on non-sunny $x=1$).
            In no case do we get two sunny lines. Contradiction.
        *   If $L_3=y=2$, it covers $\{(1,2),(2,2)\}$. The remaining points are $R_5 = \{(1,1),(2,1),(1,3),(3,1)\}$. This case is symmetric to $L_3=x=2$ by reflection across $y=x$, which preserves sunniness. No partition yields two sunny lines. Contradiction.
        *   If $L_3=x+y=3$, it covers $\{(1,2),(2,1)\}$. The remaining points are $R_6 = \{(1,1),(1,3),(2,2),(3,1)\}$. The partitions of $R_6$ into two pairs are:
            - $\{(1,1),(2,2)\}$ (on sunny $y=x$) and $\{(1,3),(3,1)\}$ (on non-sunny $x+y=4$).
            - $\{(1,1),(1,3)\}$ (on non-sunny $x=1$) and $\{(2,2),(3,1)\}$ (on non-sunny $x+y=4$).
            - $\{(1,1),(3,1)\}$ (on non-sunny $y=1$) and $\{(1,3),(2,2)\}$ (on non-sunny $x+y=4$).
            In no case do we get two sunny lines. Contradiction.
    3.  **$L_3$ covers 1 or 0 points of $P_3$**: The remaining $\ge 5$ points must be covered by two sunny lines $L_1, L_2$. By the lemma, they can cover at most $2+2=4$ points. This is not enough. Contradiction.

Since all cases lead to a contradiction, $k=2$ is impossible for $n=3$. Thus, $K_3 = \{0,1,3\}$.

**Part 2: A Preliminary Lemma**

**Lemma:** For any integer $m \ge 2$, the set of points $P_m$ cannot be covered by $m-1$ lines.
**Proof:** We proceed by induction on $m$.
*   **Base Case ($m=2$):** $P_2 = \{(1,1), (1,2), (2,1)\}$. These three points are not collinear, so they cannot be covered by $2-1=1$ line.
*   **Inductive Step:** Assume for some $m \ge 3$ that $P_{m-1}$ cannot be covered by $m-2$ lines. Suppose for contradiction that $P_m$ can be covered by a set $\mathcal{L}$ of $m-1$ lines.
    The set $P_m$ contains the $m$ points $S = \{(1,1), (1,2), \dots, (1,m)\}$, which lie on the line $x=1$. To cover these $m$ points with $m-1$ lines from $\mathcal{L}$, by the Pigeonhole Principle, at least one line in $\mathcal{L}$ must contain at least two points from $S$. Such a line must be the line $x=1$.
    Let $L_1 = (x=1) \in \mathcal{L}$. The remaining $m-2$ lines in $\mathcal{L} \setminus \{L_1\}$ must cover the remaining points $P_m \setminus S = \{(a,b) \in P_m \mid a \ge 2\}$.
    The affine transformation $T(x,y) = (x-1, y)$ maps the set $P_m \setminus S$ bijectively onto $P_{m-1}$, since $T(P_m \setminus S) = \{(a-1,b) \mid a \ge 2, b>0, a+b \le m+1\} = \{(a',b) \mid a' \ge 1, b>0, a'+1+b \le m+1\} = \{(a',b) \mid a'>0, b>0, a'+b \le m\} = P_{m-1}$. The set of transformed lines $\{T(L) \mid L \in \mathcal{L} \setminus \{L_1\}\}$ consists of $m-2$ lines that cover $P_{m-1}$. This contradicts our induction hypothesis.
    Therefore, $P_m$ cannot be covered by $m-1$ lines for any $m \ge 2$.

**Part 3: Key Lemma for $n \ge 4$**

**Lemma:** For $n \ge 4$, any set $\mathcal{L}$ of $n$ lines covering $P_n$ must contain at least one of the lines $x=1$, $y=1$, or $x+y=n+1$.
**Proof:** Assume for contradiction that $\mathcal{L}$ covers $P_n$ but contains none of these three lines.
Let $S_V = \{(1,j) \mid 1 \le j \le n\}$, $S_H = \{(i,1) \mid 1 \le i \le n\}$, and $S_D = \{(l,n+1-l) \mid 1 \le l \le n\}$. Since $x=1 \notin \mathcal{L}$, each of the $n$ lines in $\mathcal{L}$ must intersect the line $x=1$ at exactly one point, so each line covers exactly one point from $S_V$. Similarly, each line covers exactly one point from $S_H$ and one from $S_D$.
This establishes a bijection between the lines in $\mathcal{L}$ and the points in each of the sets $S_V, S_H, S_D$.

Let $L_j$ be the line covering $(1,j)$ for $j \in \{1, \dots, n\}$. Let $L_j$ intersect $S_H$ at $(\sigma(j),1)$ and $S_D$ at $(\tau(j), n+1-\tau(j))$. Since each line covers one point from each set, $\sigma$ and $\tau$ must be permutations of $\{1, \dots, n\}$.
For $j=1$, $L_1$ covers $(1,1) \in S_V \cap S_H$, so $\sigma(1)=1$. For $j=n$, $L_n$ covers $(1,n) \in S_V \cap S_D$, so $\tau(n)=1$.
For $j \in \{2, \dots, n-1\}$, the points $(1,j)$, $(\sigma(j),1)$, and $(\tau(j), n+1-\tau(j))$ are distinct and collinear. The collinearity condition implies $\tau(j) = n+1-j + \frac{(n-j)(j-1)}{\sigma(j)-j}$.

1.  **Case $j=2$ (since $n \ge 4$):** Let $i_2=\sigma(2)$ and $l_2=\tau(2)$.
    $l_2 = n-1 + \frac{n-2}{i_2-2}$. Since $\sigma$ is a permutation and $\sigma(1)=1$, $i_2 \in \{2, \dots, n\}$.
    If $i_2=2$, the line is $x+y=3$. For this line to pass through a point in $S_D$, we need $n+1=3 \implies n=2$, which contradicts $n \ge 4$. So $i_2 \ne 2$.
    Let $d=i_2-2 \in \{1, \dots, n-2\}$. For $l_2$ to be an integer, $d$ must divide $n-2$. If $d < n-2$, then $\frac{n-2}{d} > 1$, so $l_2 = n-1 + \frac{n-2}{d} > n$, which contradicts $l_2 \in \{1, \dots, n\}$. Thus, $d=n-2$, which implies $i_2=n$. This gives $l_2 = n-1 + 1 = n$. So we must have $\sigma(2)=n$ and $\tau(2)=n$.

2.  **Case $j=3$ (since $n \ge 4$):** Let $i_3=\sigma(3)$ and $l_3=\tau(3)$.
    $l_3 = n-2 + \frac{2(n-3)}{i_3-3}$. Since $\sigma$ is a permutation with $\sigma(1)=1, \sigma(2)=n$, we have $i_3 \in \{2, \dots, n-1\}$.
    If $i_3=3$, the line is $x+y=4$. For this to pass through a point in $S_D$, we need $n+1=4 \implies n=3$, which contradicts $n \ge 4$. So $i_3 \ne 3$.
    Let $d=i_3-3$. Then $i_3 \in \{2, 4, \dots, n-1\}$, so $d \in \{-1, 1, 2, \dots, n-4\}$.
    If $d>0$, then $d \le n-4$. For $l_3 \le n$, we need $n-2 + \frac{2(n-3)}{d} \le n \implies \frac{2(n-3)}{d} \le 2 \implies n-3 \le d$. This requires $n-3 \le d \le n-4$, which is impossible.
    If $d<0$, the only possibility is $d=-1$, which means $i_3=2$. Then $l_3 = n-2 + \frac{2(n-3)}{-1} = 4-n$. For $l_3 \in \{1, \dots, n\}$, we need $1 \le 4-n$, which implies $n \le 3$. This contradicts $n \ge 4$.

Since all possibilities for $\sigma(3)$ lead to a contradiction, our initial assumption must be false. Thus, for $n \ge 4$, $\mathcal{L}$ must contain one of the lines $x=1, y=1, x+y=n+1$.

**Part 4: Recurrence Relation for $K_n$**

*   **$K_n \subseteq K_{n-1}$ for $n \ge 4$:**
    Let $\mathcal{L}$ be a valid configuration of $n$ lines for $P_n$ with $k$ sunny lines. By the Key Lemma, $\mathcal{L}$ must contain a line $L \in \{x=1, y=1, x+y=n+1\}$. All these lines are non-sunny. The number of sunny lines in $\mathcal{L} \setminus \{L\}$ is $k$.
    If $L=x+y=n+1$, it covers the points $\{(i, n+1-i)\}_{i=1}^n$. The remaining $n-1$ lines must cover $P_n \setminus L = P_{n-1}$. So $\mathcal{L} \setminus \{L\}$ is a valid configuration for $P_{n-1}$ with $k$ sunny lines. Thus $k \in K_{n-1}$.
    If $L=x=1$, the remaining $n-1$ lines cover $P_n \setminus \{(1,j)\}_{j=1}^n$. The transformation $T(x,y)=(x-1,y)$ maps these points to $P_{n-1}$. The set of transformed lines is a valid configuration for $P_{n-1}$. $T$ preserves slopes, so sunniness is preserved. The number of sunny lines is $k$, so $k \in K_{n-1}$. The case $L=y=1$ is symmetric.
    In all cases, if $k \in K_n$, then $k \in K_{n-1}$.

*   **$K_{n-1} \subseteq K_n$ for $n \ge 4$:**
    Let $\mathcal{L}'$ be a valid configuration for $P_{n-1}$ with $k$ sunny lines. Let $T^{-1}(x',y')=(x'+1, y')$. Let $\mathcal{L}'' = \{T^{-1}(L') \mid L' \in \mathcal{L}'\}$. $\mathcal{L}''$ is a set of $n-1$ lines covering $T^{-1}(P_{n-1}) = \{(a,b) \in P_n \mid a \ge 2\}$. The number of sunny lines in $\mathcal{L}''$ is $k$.
    Consider the set of $n$ lines $\mathcal{L} = \mathcal{L}'' \cup \{x=1\}$. The line $x=1$ is not sunny and covers all points in $P_n$ with $x$-coordinate 1. Thus $\mathcal{L}$ covers $P_n$.
    We must check that the lines in $\mathcal{L}$ are distinct. This requires $x=1 \notin \mathcal{L}''$, which is equivalent to showing that the line $x'=0$ is not in $\mathcal{L}'$. Suppose $x'=0$ is in $\mathcal{L}'$. Since $x'=0$ contains no points of $P_{n-1}$ (where $a>0$), the remaining $(n-1)-1=n-2$ lines in $\mathcal{L}'$ must cover all of $P_{n-1}$. But by our Preliminary Lemma, for $n \ge 4$, $P_{n-1}$ cannot be covered by $n-2$ lines. This is a contradiction.
    Thus, $x=1 \notin \mathcal{L}''$, and $\mathcal{L}$ is a valid configuration for $P_n$ with $k$ sunny lines. So if $k \in K_{n-1}$, then $k \in K_n$.

**Part 5: Final Conclusion**

From Part 4, we have $K_n = K_{n-1}$ for all $n \ge 4$. By induction, this means $K_n = K_3$ for all $n \ge 3$.
In Part 1, we established that $K_3 = \{0, 1, 3\}$.
Therefore, for any integer $n \ge 3$, the set of all possible values for $k$ is $\{0, 1, 3\}$.

\subsubsection{Solution with hint}

\textit{The following solution was generated by providing the model with the additional prompt: ``Let us try to solve the problem by induction.''}

Let $S_n = \{(a,b) \in \mathbb{Z}^2 \mid a \ge 1, b \ge 1, a+b \le n+1\}$. A line is *sunny* if its slope is not $0, \infty,$ or $-1$. We seek the set of non-negative integers $k$ such that for a given integer $n \ge 3$, there exist $n$ distinct lines covering $S_n$, exactly $k$ of which are sunny.

**Part 1: Reduction to a Core Problem**

Let $\mathcal{L}$ be a set of $n$ distinct lines covering $S_n$. The *principal boundary lines* of $S_n$ are the lines $x=1$, $y=1$, and $x+y=n+1$. These are all non-sunny.

**Lemma:** If a principal boundary line $L$ of $S_n$ is in $\mathcal{L}$, then there exists a set of $n-1$ distinct lines $\mathcal{L}'$ that covers $S_{n-1}$ and has the same number of sunny lines as $\mathcal{L}$.

*Proof.* Let $k$ be the number of sunny lines in $\mathcal{L}$. The line $L$ is not sunny.
1.  Suppose $L$ is the line $x=1$. This line covers the points $\{(1,b) \mid 1 \le b \le n\}$. The remaining $n-1$ lines in $\mathcal{L} \setminus \{L\}$ must cover the set $S'_n = \{(a,b) \in S_n \mid a \ge 2\}$. The affine transformation $\phi(a,b) = (a-1, b)$ is a bijection from $S'_n$ to $S_{n-1}$. Let $\mathcal{L}'$ be the set of lines obtained by transforming the lines in $\mathcal{L} \setminus \{L\}$. A point $(x',y')$ is on a transformed line $l'$ if its pre-image under $\phi$, which is $(x'+1, y')$, is on the original line $l$. If $l$ has equation $Ax+By+C=0$, the transformed line $l'$ has equation $A(x+1)+By+C=0$. This transformation preserves the slope of any line, so the number of sunny lines in $\mathcal{L}'$ is $k$. The lines in $\mathcal{L}'$ are distinct: if $l_1: A_1x+B_1y+C_1=0$ and $l_2: A_2x+B_2y+C_2=0$ are distinct lines in $\mathcal{L} \setminus \{L\}$, their images are $l'_1: A_1x+B_1y+(A_1+C_1)=0$ and $l'_2: A_2x+B_2y+(A_2+C_2)=0$. If $l'_1=l'_2$, their coefficients are proportional, so $(A_1, B_1, A_1+C_1) = \lambda(A_2, B_2, A_2+C_2)$ for some $\lambda \ne 0$. This implies $A_1=\lambda A_2$, $B_1=\lambda B_2$, and $A_1+C_1=\lambda(A_2+C_2)$. Substituting the first into the third gives $\lambda A_2+C_1=\lambda A_2+\lambda C_2$, so $C_1=\lambda C_2$. Thus $(A_1,B_1,C_1)=\lambda(A_2,B_2,C_2)$, contradicting the distinctness of $l_1, l_2$.
2.  If $L$ is $y=1$, a symmetric argument with the transformation $(a,b) \mapsto (a, b-1)$ applies.
3.  If $L$ is $x+y=n+1$, it covers the points $\{(a,b) \in S_n \mid a+b=n+1\}$. The other $n-1$ lines must cover the remaining points, which form the set $\{(a,b) \in S_n \mid a+b \le n\} = S_{n-1}$. Thus, $\mathcal{L}' = \mathcal{L} \setminus \{L\}$ is a set of $n-1$ distinct lines covering $S_{n-1}$ with $k$ sunny lines.

By repeatedly applying this lemma, any configuration for $S_n$ can be reduced. This process terminates when we obtain a set of $m$ lines covering $S_m$ (for some $m \le n$) where none of the lines are principal boundary lines of $S_m$. The number of sunny lines in this new configuration is still $k$.

Let $\mathcal{L}_m$ be such a set of $m$ lines covering $S_m$.
Let $C_{m,x} = \{(1,b) \mid 1 \le b \le m\}$. These $m$ points must be covered by $\mathcal{L}_m$. Since $x=1 \notin \mathcal{L}_m$, any vertical line in $\mathcal{L}_m$ (of the form $x=c, c \ne 1$) misses all points in $C_{m,x}$. Each of the non-vertical lines in $\mathcal{L}_m$ can cover at most one point of $C_{m,x}$. If there are $v$ vertical lines in $\mathcal{L}_m$, then the remaining $m-v$ lines must cover all $m$ points, so $m-v \ge m$, which implies $v=0$. Thus, $\mathcal{L}_m$ contains no vertical lines (slope $\infty$).
By a symmetric argument considering $C_{m,y} = \{(a,1) \mid 1 \le a \le m\}$, we deduce $\mathcal{L}_m$ contains no horizontal lines (slope $0$).
By considering $C_{m,sum} = \{(a,b) \in S_m \mid a+b=m+1\}$, we deduce $\mathcal{L}_m$ contains no lines of slope $-1$.
Therefore, all $m$ lines in $\mathcal{L}_m$ must be sunny. This implies $k=m$. The problem reduces to finding for which non-negative integers $k$ it is possible to cover $S_k$ with $k$ distinct sunny lines, none of which are principal boundary lines of $S_k$.

**Part 2: Analysis of the Core Problem**

*   **$k=0$:** $S_0 = \emptyset$. This is covered by 0 lines. So $k=0$ is possible.
*   **$k=1$:** $S_1 = \{(1,1)\}$. The line $y=x$ is sunny, covers $(1,1)$, and is not a principal boundary line of $S_1$ (which are $x=1, y=1, x+y=2$). So $k=1$ is possible.
*   **$k=2$:** $S_2 = \{(1,1), (1,2), (2,1)\}$. We need to cover these three points with two distinct sunny lines, neither of which can be a principal boundary line of $S_2$ ($x=1, y=1, x+y=3$). A line passing through any two of the points in $S_2$ must be one of these three principal boundary lines, which are non-sunny. Therefore, a sunny line can pass through at most one point of $S_2$. Consequently, two sunny lines can cover at most two points, leaving at least one point of $S_2$ uncovered. So $k=2$ is impossible.
*   **$k=3$:** $S_3 = \{(1,1), (1,2), (1,3), (2,1), (2,2), (3,1)\}$. The following three lines are sunny, distinct, not principal boundary lines of $S_3$, and cover $S_3$:
    $L_1: y = -2x+5$ (covers $(1,3), (2,1)$)
    $L_2: y = x$ (covers $(1,1), (2,2)$)
    $L_3: x+2y=5$ (covers $(1,2), (3,1)$)
    So $k=3$ is possible.
*   **$k \ge 4$:** Assume there exists a set $\mathcal{L}$ of $k$ distinct sunny lines covering $S_k$, with no principal boundary lines. As argued, each line in $\mathcal{L}$ must intersect each of the sets $C_y=\{(i,1)\}_{i=1}^k$, $C_x=\{(1,j)\}_{j=1}^k$, and $C_s=\{(p,k+1-p)\}_{p=1}^k$ at exactly one point.
    We can label the lines $L_i$ for $i \in \{1, \dots, k\}$ such that $L_i$ is the unique line in $\mathcal{L}$ passing through $(i,1)$. For each $i$, $L_i$ must also pass through a unique point $(1, \sigma(i)) \in C_x$ and a unique point $(\pi(i), k+1-\pi(i)) \in C_s$. The maps $\sigma, \pi: \{1, \dots, k\} \to \{1, \dots, k\}$ must be permutations.

    These permutations must satisfy several properties:
    1.  $L_1$ passes through $(1,1)$, so $\sigma(1)=1$.
    2.  $L_k$ passes through $(k,1)$, which is also in $C_s$. Thus, $(\pi(k), k+1-\pi(k))=(k,1)$, which implies $\pi(k)=k$.
    3.  There is a unique line $L_j$ passing through $(1,k) \in C_x$, so $\sigma(j)=k$. The point $(1,k)$ is also in $C_s$, so for $L_j$, $(\pi(j), k+1-\pi(j))=(1,k)$, which implies $\pi(j)=1$. Since $\sigma(1)=1 \ne k$ and $\pi(k)=k \ne 1$, we have $j \ne 1,k$, so $j \in \{2, \dots, k-1\}$.
    4.  For $i \in \{2, \dots, k\}$, if $\sigma(i)=i$, $L_i$ would pass through $(i,1)$ and $(1,i)$, giving it slope $-1$, which is not allowed. So $\sigma(i) \ne i$ for $i \ge 2$.
    5.  For $i \in \{1, \dots, k-1\}$, if $\pi(i)=i$, $L_i$ would pass through $(i,1)$ and $(i, k+1-i)$, making it a vertical line, which is not allowed. So $\pi(i) \ne i$ for $i \le k-1$.

    For any $i \in \{1, \dots, k\}$, the three points $P_1=(i,1)$, $P_2=(1, \sigma(i))$, and $P_3=(\pi(i), k+1-\pi(i))$ must be collinear. For $i \in \{2, \dots, k-1\} \setminus \{j\}$, these three points are distinct. To prove this:
    - $P_1=P_2 \implies i=1$, but $i \ge 2$.
    - $P_1=P_3 \implies \pi(i)=i$, but $\pi(i) \ne i$ for $i \le k-1$.
    - $P_2=P_3 \implies \pi(i)=1$ and $\sigma(i)=k$. By definition of $j$, this means $i=j$. But we consider $i \ne j$.
    Thus, for $i \in \{2, \dots, k-1\} \setminus \{j\}$, the points are distinct. Collinearity implies their slopes are equal: $\frac{\sigma(i)-1}{1-i} = \frac{k+1-\pi(i)-1}{\pi(i)-i}$. This gives $\sigma(i) = 1 + (i-1)\frac{k-\pi(i)}{i-\pi(i)}$.

    **Step 1: Show $j=k-1$.**
    Assume for contradiction that $j \ne k-1$. Then the formula for $\sigma(i)$ is valid for $i=k-1$.
    $\sigma(k-1) = 1 + (k-2)\frac{k-\pi(k-1)}{k-1-\pi(k-1)}$.
    From the properties of $\pi$: $\pi(k-1) \in \{1, \dots, k\}$, $\pi(k-1) \ne \pi(j)=1$, $\pi(k-1) \ne \pi(k)=k$, and $\pi(k-1) \ne k-1$. So $\pi(k-1) \in \{2, \dots, k-2\}$.
    Let $d = k-1-\pi(k-1)$. Then $d \in \{1, \dots, k-3\}$.
    $\sigma(k-1) = 1 + (k-2)\frac{k-(k-1-d)}{d} = 1 + (k-2)\frac{d+1}{d} = 1+(k-2)(1+\frac{1}{d}) = k-1 + \frac{k-2}{d}$.
    Since $k \ge 4$, we have $d \le k-3$, so $\frac{k-2}{d} \ge \frac{k-2}{k-3} = 1+\frac{1}{k-3} > 1$.
    So $\sigma(k-1) > k-1+1=k$. This contradicts $\sigma(k-1) \in \{1, \dots, k\}$.
    Thus, our assumption was false. We must have $j=k-1$.

    **Step 2: Derive a final contradiction.**
    We have established that for $k \ge 4$, it must be that $j=k-1$, which means $\sigma(k-1)=k$ and $\pi(k-1)=1$.
    Consider $i=2$. Since $k \ge 4$, $j=k-1 \ge 3$, so $j \ne 2$. The formula for $\sigma(i)$ is valid for $i=2$:
    $\sigma(2) = 1 + \frac{k-\pi(2)}{2-\pi(2)}$.
    Since $\pi$ is a permutation, its image on $\{1, \dots, k-2\}$ is $\{1, \dots, k\} \setminus \{\pi(k-1), \pi(k)\}$. With $\pi(k-1)=1$ and $\pi(k)=k$, we have $\pi(\{1, \dots, k-2\}) = \{2, \dots, k-1\}$.
    So $\pi(2) \in \{2, \dots, k-1\}$. By property 5, $\pi(2) \ne 2$. Thus $\pi(2) \in \{3, \dots, k-1\}$.
    Let $d' = \pi(2)-2$. Then $d' \in \{1, \dots, k-3\}$.
    The denominator is $2-\pi(2) = -d'$.
    $\sigma(2) = 1 + \frac{k-(d'+2)}{-d'} = 1 - \frac{k-d'-2}{d'} = 1 - \left(\frac{k-2}{d'}-1\right) = 2 - \frac{k-2}{d'}$.
    Since $d' \le k-3$, we have $\frac{k-2}{d'} \ge \frac{k-2}{k-3} = 1+\frac{1}{k-3} > 1$ for $k \ge 4$.
    So $\sigma(2) = 2 - \frac{k-2}{d'} < 2-1=1$.
    This contradicts $\sigma(2) \in \{1, \dots, k\}$.
    This final contradiction shows that no such configuration is possible for $k \ge 4$.

**Part 3: Constructions for general $n \ge 3$**

The set of possible values for $k$ is $\{0, 1, 3\}$. We now show these are all possible for any $n \ge 3$.

*   **$k=0$:** Let $\mathcal{L} = \{x=i \mid i=1, 2, \dots, n\}$. These are $n$ distinct non-sunny lines. Any point $(a,b) \in S_n$ has $1 \le a \le n$, so it lies on the line $x=a \in \mathcal{L}$.
*   **$k=1$:** Let $\mathcal{L} = \{x=i \mid i=1, \dots, n-1\} \cup \{y=x-(n-1)\}$. The first $n-1$ lines are non-sunny and cover all points $(a,b) \in S_n$ with $a \le n-1$. The only points in $S_n$ not covered are those with $a \ge n$. If $(a,b) \in S_n$ and $a \ge n$, then $n+b \le a+b \le n+1$, which implies $b \le 1$. Since $b \ge 1$, we must have $b=1$. Then $a+1 \le n+1 \implies a \le n$. The only uncovered point is $(n,1)$. The line $y=x-(n-1)$ has slope 1 (sunny), is distinct from the other lines, and passes through $(n,1)$ since $1 = n-(n-1)$.
*   **$k=3$:** Let $\mathcal{L}_{ns} = \{x+y=c \mid c=5, 6, \dots, n+1\}$. This is a set of $n-3$ distinct non-sunny lines (if $n=3$, this set is empty). These lines cover all points $(a,b) \in S_n$ with $a+b \ge 5$. The remaining points to be covered are $U = \{(a,b) \in S_n \mid a+b \le 4\}$. This set is precisely $S_3$. As shown in Part 2, $S_3$ can be covered by 3 distinct sunny lines. Let $\mathcal{L}_s$ be this set of 3 lines. Then $\mathcal{L} = \mathcal{L}_{ns} \cup \mathcal{L}_s$ is a set of $(n-3)+3=n$ lines covering $S_n$. The lines in $\mathcal{L}_s$ have slopes different from $-1$, so they are distinct from lines in $\mathcal{L}_{ns}$. This construction provides exactly 3 sunny lines.

Thus, for any integer $n \ge 3$, configurations with $k \in \{0, 1, 3\}$ sunny lines are possible.

\subsection{Problem 2}

\begin{problem}
Let $\Omega$ and $\Gamma$ be circles with centers $M$ and $N$, respectively, such that the radius of $\Omega$ is less than the radius of $\Gamma$. Suppose circles $\Omega$ and $\Gamma$ intersect at two distinct points $A$ and $B$. Let $MN$ intersect $\Omega$ at $C$ and $\Gamma$ at $D$, such that points $C$, $M$, $N$, and $D$ lie on the line in that order. Let $P$ be the circumcenter of triangle $ACD$. Line $AP$ intersects $\Omega$ again at $E \ne A$. Line $AP$ intersects $\Gamma$ again at $F \ne A$. Let $H$ be the orthocenter of triangle $PMN$.

Prove that the line through $H$ parallel to $AP$ is tangent to the circumcircle of triangle $BEF$.

(The orthocenter of a triangle is the point of intersection of its altitudes.)
\end{problem}

\subsubsection{Proof without hint}

Let the line containing the centers $M$ and $N$ be the x-axis. The radical axis of circles $\Omega$ and $\Gamma$ is the line $AB$, which is perpendicular to the line of centers. We place the origin $O(0,0)$ at the intersection of $MN$ and $AB$.

Let $M=(m,0)$ and $N=(n,0)$. The points $A$ and $B$ have coordinates $(0,a)$ and $(0,-a)$ for some $a>0$. Since $A$ lies on both circles:
$MA^2 = (0-m)^2 + (a-0)^2 = r^2 \implies m^2+a^2=r^2$.
$NA^2 = (0-n)^2 + (a-0)^2 = R^2 \implies n^2+a^2=R^2$.
This gives the key relation $a^2 = r^2-m^2 = R^2-n^2$, which implies $n^2-m^2=R^2-r^2$.
The problem states that $C, M, N, D$ lie on the line in that order, so $x_C < x_M < x_N < x_D$.
$C$ is on $\Omega$ and the x-axis, so its x-coordinate is $m-r$. Thus $C=(m-r,0)$.
$D$ is on $\Gamma$ and the x-axis, so its x-coordinate is $n+R$. Thus $D=(n+R,0)$.

**1. Coordinates of P and H**
$P(x_P, y_P)$ is the circumcenter of $\triangle ACD$. It lies on the perpendicular bisector of $CD$, so its x-coordinate is $x_P = \frac{(m-r)+(n+R)}{2}$.
The condition $PA^2=PC^2$ gives $x_P^2 + (y_P-a)^2 = (x_P-(m-r))^2 + y_P^2$, which simplifies to $x_P^2 + y_P^2 - 2ay_P + a^2 = x_P^2 - 2x_P(m-r) + (m-r)^2 + y_P^2$. This yields $2ay_P = 2x_P(m-r) - (m-r)^2 + a^2$. Substituting the expression for $x_P$:
$2ay_P = (m-r+n+R)(m-r) - (m-r)^2 + a^2 = (m-r)(n+R)+a^2$.

$H(x_H, y_H)$ is the orthocenter of $\triangle PMN$. The altitude from $P$ to $MN$ (the x-axis) is the vertical line $x=x_P$, so $x_H=x_P$.
The altitude from $M$ in $\triangle PMN$ is perpendicular to the line $PN$.

*   **Lemma:** The line $PN$ is perpendicular to the line $AD$.
*   **Proof:** $P$ is the circumcenter of $\triangle ACD$, so by definition, $P$ is equidistant from its vertices. Thus $PA=PD$. This means $P$ lies on the perpendicular bisector of segment $AD$.
    $N$ is the center of circle $\Gamma$. Points $A=(0,a)$ and $D=(n+R,0)$ are on $\Gamma$. The radius of $\Gamma$ is $R$. We have $NA^2 = (0-n)^2+(a-0)^2 = n^2+a^2=R^2$, so $NA=R$. The distance $ND$ is $|(n+R)-n|=R$. Thus $NA=ND=R$. This means $N$ also lies on the perpendicular bisector of segment $AD$.
    Since both $P$ and $N$ lie on the perpendicular bisector of $AD$, the line $PN$ is the perpendicular bisector of $AD$. Thus, $PN \perp AD$.

Since the altitude from $M$ in $\triangle PMN$ is perpendicular to $PN$, and we have shown $PN \perp AD$, this altitude must be parallel to $AD$.
The slope of $AD$ is $k_{AD} = \frac{a-0}{0-(n+R)} = \frac{-a}{n+R}$. The slope of the line $MH$ is $k_{MH} = \frac{y_H-0}{x_H-m} = \frac{y_H}{x_P-m}$.
Equating the slopes gives $y_H = k_{AD}(x_P-m) = -\frac{a(x_P-m)}{n+R}$.

**2. The Tangency Condition**
Let $\omega$ be the circumcircle of $\triangle BEF$ with center $O_\omega(x_c, y_c)$ and radius $R_\omega$. Let $k_{AP}$ be the slope of line $AP$. The line $\ell$ through $H$ parallel to $AP$ has equation $y-y_H = k_{AP}(x-x_P)$. The condition for $\ell$ to be tangent to $\omega$ is that the square of the distance from $O_\omega$ to $\ell$ equals $R_\omega^2$:
$$ \frac{(k_{AP}(x_c-x_P) - (y_c-y_H))^2}{1+k_{AP}^2} = R_\omega^2 $$
Since $B(0,-a)$ is on $\omega$, $R_\omega^2 = (x_c-0)^2+(y_c-(-a))^2 = x_c^2+(y_c+a)^2$.
Let $T_1 = k_{AP}(x_c-x_P) - (y_c-y_H)$ and $T_2 = (1+k_{AP}^2)(x_c^2 + (y_c+a)^2)$. We need to prove $T_1^2 = T_2$.

**3. Coordinates of $E, F$ and $O_\omega$**
The line $AP$ passes through $A(0,a)$ and has slope $k_{AP}$. Its equation is $y=k_{AP}x+a$.
$E$ is the second intersection of $y=k_{AP}x+a$ with $\Omega: (x-m)^2+y^2=r^2$. Substituting $y$ gives $(x-m)^2+(k_{AP}x+a)^2=r^2$, which simplifies to $(1+k_{AP}^2)x^2+2(ak_{AP}-m)x+(m^2+a^2-r^2)=0$. Since $m^2+a^2=r^2$, this becomes $x((1+k_{AP}^2)x+2(ak_{AP}-m))=0$. The solutions are $x=0$ (for point $A$) and $x_E = -\frac{2(ak_{AP}-m)}{1+k_{AP}^2}$.
Similarly, for $F$ on $\Gamma: (x-n)^2+y^2=R^2$, we get $x_F = -\frac{2(ak_{AP}-n)}{1+k_{AP}^2}$.

*   **Lemma:** The coordinates of $O_\omega(x_c, y_c)$ satisfy:
    a) $x_c+k_{AP}y_c = m+n-ak_{AP}$
    b) $y_c = -\frac{(ak_{AP}-m)(ak_{AP}-n)}{a(1+k_{AP}^2)}$
*   **Proof:**
    a) $O_\omega$ lies on the perpendicular bisector of $EF$. The midpoint of $EF$ is $(\frac{x_E+x_F}{2}, \frac{y_E+y_F}{2})$. The slope of $EF$ (line $AP$) is $k_{AP}$. The equation of the perpendicular bisector is $y-\frac{y_E+y_F}{2} = -\frac{1}{k_{AP}}(x-\frac{x_E+x_F}{2})$. Since $O_\omega(x_c, y_c)$ is on this line, we have $x_c+k_{AP}y_c = \frac{x_E+x_F}{2}+k_{AP}\frac{y_E+y_F}{2}$. Using $y_E=k_{AP}x_E+a$ and $y_F=k_{AP}x_F+a$, the right side is $\frac{1+k_{AP}^2}{2}(x_E+x_F)+ak_{AP}$. Substituting the expressions for $x_E$ and $x_F$:
    $\frac{1+k_{AP}^2}{2} \left(-\frac{2(ak_{AP}-m)+2(ak_{AP}-n)}{1+k_{AP}^2}\right) + ak_{AP} = -(2ak_{AP}-m-n)+ak_{AP} = m+n-ak_{AP}$.
    So, $x_c+k_{AP}y_c = m+n-ak_{AP}$.

    b) $O_\omega$ is equidistant from $B(0,-a)$ and $E(x_E, y_E)$. So $O_\omega B^2 = O_\omega E^2$.
    $x_c^2+(y_c+a)^2 = (x_c-x_E)^2+(y_c-y_E)^2$.
    $2ay_c + a^2 = -2x_cx_E + x_E^2 - 2y_cy_E + y_E^2$.
    Using $y_E=k_{AP}x_E+a$:
    $2ay_c + a^2 = -2x_cx_E + x_E^2 - 2y_c(k_{AP}x_E+a) + (k_{AP}x_E+a)^2$.
    $4ay_c = -2x_E(x_c+k_{AP}y_c) + (1+k_{AP}^2)x_E^2 + 2ak_{AP}x_E$.
    Dividing by $x_E \neq 0$: $\frac{4ay_c}{x_E} = -2(x_c+k_{AP}y_c) + (1+k_{AP}^2)x_E + 2ak_{AP}$.
    Using part (a) and the expression for $x_E$:
    $\frac{4ay_c}{x_E} = -2(m+n-ak_{AP}) -2(ak_{AP}-m) + 2ak_{AP} = -2m-2n+2ak_{AP} -2ak_{AP}+2m + 2ak_{AP} = 2(ak_{AP}-n)$.
    $2ay_c = x_E(ak_{AP}-n) = \left(-\frac{2(ak_{AP}-m)}{1+k_{AP}^2}\right)(ak_{AP}-n)$.
    $y_c = -\frac{(ak_{AP}-m)(ak_{AP}-n)}{a(1+k_{AP}^2)}$. This completes the proof of the lemma.

**4. Algebraic Verification**

*   **Lemma:** $T_1 = k_{AP}(x_c-x_P) - (y_c-y_H) = \frac{rR}{a}$.
*   **Proof:** $T_1 = (k_{AP}x_c-y_c) - (k_{AP}x_P-y_H)$.
    From part (a) of the previous lemma, $x_c = m+n-ak_{AP}-k_{AP}y_c$.
    $k_{AP}x_c-y_c = k_{AP}(m+n-ak_{AP})-(1+k_{AP}^2)y_c$. Substituting $y_c$:
    $k_{AP}(m+n-ak_{AP}) + \frac{(ak_{AP}-m)(ak_{AP}-n)}{a} = \frac{ak_{AP}(m+n)-a^2k_{AP}^2+a^2k_{AP}^2-ak_{AP}(m+n)+mn}{a} = \frac{mn}{a}$.
    The slope of $AP$ is $k_{AP} = \frac{y_P-a}{x_P}$, so $k_{AP}x_P = y_P-a$. Thus $k_{AP}x_P-y_H = y_P-a-y_H$.
    $T_1 = \frac{mn}{a} - (y_P-a-y_H) = \frac{mn - a(y_P-a-y_H)}{a}$.
    $2aT_1 = 2mn - 2ay_P + 2a^2 + 2ay_H$.
    Using $2ay_P = (m-r)(n+R)+a^2$ and $2ay_H = -2a^2\frac{x_P-m}{n+R} = -a^2\frac{n-m-r+R}{n+R}$:
    $2aT_1 = 2mn - (mn+mR-rn-rR+a^2) + 2a^2 - a^2\frac{n-m-r+R}{n+R}$.
    $2aT_1 = mn - mR + rn + rR + a^2 - (R^2-n^2)\frac{n-m-r+R}{n+R}$ since $a^2=R^2-n^2$.
    $2aT_1 = mn - mR + rn + rR + (R-n)(R+n) - (R-n)(n-m-r+R)$.
    $2aT_1 = mn - mR + rn + rR + (R-n)[(R+n)-(n-m-r+R)]$.
    The term in brackets is $R+n-n+m+r-R = m+r$.
    $2aT_1 = mn - mR + rn + rR + (R-n)(m+r) = mn - mR + rn + rR + Rm + Rr - nm - nr = 2rR$.
    Thus, $T_1 = \frac{rR}{a}$.

*   **Lemma:** $T_2 = (1+k_{AP}^2)(x_c^2 + (y_c+a)^2) = \frac{r^2R^2}{a^2}$.
*   **Proof:** From the relations for $x_c, y_c$, we have $x_c = \frac{a(m+n)+k_{AP}(mn-a^2)}{a(1+k_{AP}^2)}$ and $y_c+a = \frac{a^2-mn+ak_{AP}(m+n)}{a(1+k_{AP}^2)}$.
    Let $U=a(m+n)$ and $V=mn-a^2$. Then $a(1+k_{AP}^2)x_c = U+k_{AP}V$ and $a(1+k_{AP}^2)(y_c+a) = -V+k_{AP}U$.
    $a^2(1+k_{AP}^2)^2(x_c^2+(y_c+a)^2) = (U+k_{AP}V)^2 + (-V+k_{AP}U)^2 = (U^2+V^2)(1+k_{AP}^2)$.
    So, $T_2 = (1+k_{AP}^2)(x_c^2+(y_c+a)^2) = \frac{U^2+V^2}{a^2} = \frac{(a(m+n))^2+(mn-a^2)^2}{a^2}$.
    We verify that the numerator is $r^2R^2$:
    $r^2R^2 = (m^2+a^2)(n^2+a^2) = m^2n^2+a^2(m^2+n^2)+a^4$.
    $(a(m+n))^2+(mn-a^2)^2 = a^2(m^2+2mn+n^2) + (m^2n^2-2mna^2+a^4) = m^2n^2+a^2(m^2+n^2)+a^4$.
    The expressions are identical. So, $T_2 = \frac{r^2R^2}{a^2}$.

**Conclusion**
From our lemmas, $T_1^2 = (\frac{rR}{a})^2 = \frac{r^2R^2}{a^2}$ and $T_2 = \frac{r^2R^2}{a^2}$.
Since $T_1^2 = T_2$, the tangency condition is satisfied. The line through $H$ parallel to $AP$ is tangent to the circumcircle of $\triangle BEF$.

\subsubsection{Proof with hint}

\textit{The following proof was generated by providing the model with the additional prompt: ``Let us try to solve the problem by analytic geometry.''}

**1. Coordinate System and Geometric Setup**

We establish a Cartesian coordinate system. Let the line of centers $MN$ be the $x$-axis, and the common chord $AB$ be the $y$-axis. Let their intersection be the origin $O(0,0)$.

*   The points of intersection of the circles are $A(0,a)$ and $B(0,-a)$ for some $a>0$.
*   The centers of the circles are $M(m,0)$ and $N(n,0)$.
*   Circle $\Omega$ has center $M(m,0)$ and passes through $A(0,a)$. Its radius is $R_\Omega = \sqrt{m^2+a^2}$. The equation of $\Omega$ is $(x-m)^2+y^2 = m^2+a^2$, which simplifies to $x^2-2mx+y^2=a^2$.
*   Circle $\Gamma$ has center $N(n,0)$ and passes through $A(0,a)$. Its radius is $R_\Gamma = \sqrt{n^2+a^2}$. The equation of $\Gamma$ is $(x-n)^2+y^2 = n^2+a^2$, which simplifies to $x^2-2nx+y^2=a^2$.
*   The line $MN$ is the $x$-axis. The points $C$ and $D$ lie on the $x$-axis. Let their coordinates be $C(c,0)$ and $D(d,0)$.
*   The condition that $C, M, N, D$ lie on the line in that order implies their $x$-coordinates are ordered: $c < m < n < d$.
*   $C$ is an intersection of $\Omega$ with the $x$-axis. Its coordinate $c$ is a root of $x^2-2mx-a^2=0$. The roots are $m \pm \sqrt{m^2+a^2}$. Since $c<m$, we must have $c = m-\sqrt{m^2+a^2}$.
*   $D$ is an intersection of $\Gamma$ with the $x$-axis. Its coordinate $d$ is a root of $x^2-2nx-a^2=0$. The roots are $n \pm \sqrt{n^2+a^2}$. Since $d>n$, we must have $d=n+\sqrt{n^2+a^2}$.
*   From the equations for $c$ and $d$, we have $c^2-2mc-a^2=0$ and $d^2-2nd-a^2=0$. Since $c<0$ and $d>0$, $c,d \neq 0$. We can express $m$ and $n$ as:
    $m = \frac{c^2-a^2}{2c}$ and $n = \frac{d^2-a^2}{2d}$.

**2. Preliminary Geometric and Algebraic Properties**

We establish two key properties derived from the problem's conditions.

*   **Lemma 1:** $c+d \neq 0$.
    *   **Proof:** The condition $R_\Omega < R_\Gamma$ implies $m^2+a^2 < n^2+a^2$, so $m^2 < n^2$.
    *   The ordering $C,M,N,D$ implies $m<n$.
    *   From $m<n$ and $m^2<n^2$, we have $n^2-m^2 > 0$, so $(n-m)(n+m)>0$. Since $n-m>0$, we must have $n+m>0$.
    *   Also, $m^2 < n^2 \implies m^2+a^2 < n^2+a^2 \implies \sqrt{m^2+a^2} < \sqrt{n^2+a^2}$.
    *   Assume for contradiction that $c+d=0$. Using the expressions for $c$ and $d$:
        $(m-\sqrt{m^2+a^2}) + (n+\sqrt{n^2+a^2}) = 0 \implies m+n = \sqrt{m^2+a^2} - \sqrt{n^2+a^2}$.
    *   The left side, $m+n$, is positive. The right side, $\sqrt{m^2+a^2} - \sqrt{n^2+a^2}$, is negative. This is a contradiction.
    *   Thus, our assumption is false, and $c+d \neq 0$.

*   **Lemma 2:** $a^2+cd < 0$.
    *   **Proof:** The condition $m<n$ implies $\frac{c^2-a^2}{2c} < \frac{d^2-a^2}{2d}$.
    *   From their definitions, $c<0$ and $d>0$, so $cd<0$. Multiplying the inequality by $2cd$ (which is negative) reverses the inequality sign:
        $d(c^2-a^2) > c(d^2-a^2) \implies c^2d-a^2d > cd^2-a^2c \implies cd(c-d) > -a^2(c-d)$.
    *   Since $c<d$, we have $c-d<0$. Dividing by $c-d$ reverses the inequality sign again:
        $cd < -a^2 \implies a^2+cd < 0$.

**3. Coordinates of P and H**

*   Let $P(x_P, y_P)$ be the circumcenter of $\triangle ACD$ with vertices $A(0,a)$, $C(c,0)$, $D(d,0)$.
*   $P$ lies on the perpendicular bisector of segment $CD$, which is the line $x = \frac{c+d}{2}$. So, $x_P = \frac{c+d}{2}$.
*   The condition $PA^2 = PC^2$ gives $(x_P-0)^2 + (y_P-a)^2 = (x_P-c)^2 + (y_P-0)^2$.
    $x_P^2 + y_P^2 - 2ay_P + a^2 = x_P^2 - 2cx_P + c^2 + y_P^2 \implies -2ay_P + a^2 = -2cx_P + c^2$.
    Substituting $x_P = \frac{c+d}{2}$: $-2ay_P + a^2 = -c(c+d) + c^2 = -cd \implies y_P = \frac{a^2+cd}{2a}$.
    So, $P\left(\frac{c+d}{2}, \frac{a^2+cd}{2a}\right)$.

*   Let $H(x_H, y_H)$ be the orthocenter of $\triangle PMN$ with vertices $P(x_P, y_P)$, $M(m,0)$, $N(n,0)$.
*   The altitude from $P$ to $MN$ (on the $x$-axis) is the line $x=x_P$. Thus, $x_H = x_P = \frac{c+d}{2}$.
*   The altitude from $M$ is perpendicular to $PN$. The slope of this altitude is $-\frac{x_P-n}{y_P}$. The line is $y-0 = -\frac{x_P-n}{y_P}(x-m)$.
*   $H$ lies on this line, so $y_H = -\frac{x_P-n}{y_P}(x_H-m) = -\frac{(x_P-m)(x_P-n)}{y_P}$.
*   We express the numerator in terms of $a,c,d$:
    $x_P-m = \frac{c+d}{2} - \frac{c^2-a^2}{2c} = \frac{c(c+d)-(c^2-a^2)}{2c} = \frac{cd+a^2}{2c}$.
    $x_P-n = \frac{c+d}{2} - \frac{d^2-a^2}{2d} = \frac{d(c+d)-(d^2-a^2)}{2d} = \frac{cd+a^2}{2d}$.
*   Substituting these into the expression for $y_H$:
    $y_H = -\frac{1}{y_P} \left(\frac{cd+a^2}{2c}\right) \left(\frac{cd+a^2}{2d}\right) = -\frac{(cd+a^2)^2}{4cd y_P}$.
    Using $y_P = \frac{a^2+cd}{2a}$ and $a^2+cd \neq 0$ (from Lemma 2), we get:
    $y_H = -\frac{(cd+a^2)^2}{4cd} \frac{2a}{a^2+cd} = -\frac{a(a^2+cd)}{2cd}$.
*   So, the coordinates of the orthocenter are $H\left(\frac{c+d}{2}, -\frac{a(a^2+cd)}{2cd}\right)$.

**4. The Line AP and its Intersections**

*   Since $a \neq 0$ and $c+d \neq 0$ (Lemma 1), the slope of line $AP$, denoted $k_{AP}$, is well-defined:
    $k_{AP} = \frac{y_P-a}{x_P-0} = \frac{\frac{a^2+cd}{2a}-a}{\frac{c+d}{2}} = \frac{a^2+cd-2a^2}{a(c+d)} = \frac{cd-a^2}{a(c+d)}$.
*   The line $AP$ has equation $y=k_{AP}x+a$. $E(x_E, y_E)$ and $F(x_F, y_F)$ are the other intersection points of this line with $\Omega$ and $\Gamma$ respectively.
*   To find $E$, substitute $y=k_{AP}x+a$ into the equation of $\Omega$, $x^2-2mx+y^2=a^2$:
    $x^2-2mx+(k_{AP}x+a)^2=a^2 \implies x((1+k_{AP}^2)x - 2m + 2ak_{AP})=0$.
    The roots are $x=0$ (for point $A$) and $x_E = \frac{2(m-ak_{AP})}{1+k_{AP}^2}$.
*   Similarly, for $F$ on $\Gamma$ ($x^2-2nx+y^2=a^2$): $x_F = \frac{2(n-ak_{AP})}{1+k_{AP}^2}$.

**5. Auxiliary Algebraic Identities**

We derive identities that will simplify subsequent calculations.
*   **Identity 1:** $m-ak_{AP} = \frac{c^2-a^2}{2c} - a\left(\frac{cd-a^2}{a(c+d)}\right) = \frac{(c^2-a^2)(c+d) - 2c(cd-a^2)}{2c(c+d)} = \frac{(c^2+a^2)(c-d)}{2c(c+d)}$.
*   **Identity 2:** $n-ak_{AP} = \frac{d^2-a^2}{2d} - \frac{cd-a^2}{c+d} = \frac{(d^2-a^2)(c+d) - 2d(cd-a^2)}{2d(c+d)} = \frac{(d^2+a^2)(d-c)}{2d(c+d)}$.
*   **Identity 3:** $1+k_{AP}^2 = 1+\left(\frac{cd-a^2}{a(c+d)}\right)^2 = \frac{a^2(c+d)^2+(cd-a^2)^2}{a^2(c+d)^2} = \frac{(c^2+a^2)(d^2+a^2)}{a^2(c+d)^2}$.
*   **Identity 4:** $k_{AP}+\frac{2a}{x_E} = \frac{cd-a^2}{a(c+d)} + \frac{a(1+k_{AP}^2)}{m-ak_{AP}} = \frac{cd-a^2}{a(c+d)} + a \frac{(c^2+a^2)(d^2+a^2)}{a^2(c+d)^2} \left/ \frac{(c^2+a^2)(c-d)}{2c(c+d)} \right. = \frac{cd-a^2}{a(c+d)} + \frac{2c(d^2+a^2)}{a(c+d)(c-d)} = \frac{(cd-a^2)(c-d)+2c(d^2+a^2)}{a(c+d)(c-d)} = \frac{cd(c+d)+a^2(c+d)}{a(c+d)(c-d)} = \frac{cd+a^2}{a(c-d)}$.

**6. The Circumcircle of $\triangle BEF$**

Let $K(x_K, y_K)$ be the circumcenter of $\triangle BEF$. $K$ lies on the perpendicular bisectors of $BE$ and $BF$.
The perpendicular bisector of $BE$ is given by $KB^2=KE^2$, which simplifies to $2x_Kx_E+2y_K(y_E+a) = x_E^2+y_E^2-a^2+2a^2 = 2mx_E+2a^2$. Using $y_E=k_{AP}x_E+a$, we get $x_Kx_E+y_K(k_{AP}x_E+2a) = mx_E+a^2$. This is incorrect.
Let's re-derive. $KB^2=KE^2 \implies x_K^2+(y_K+a)^2 = (x_K-x_E)^2+(y_K-y_E)^2$.
$x_K^2+y_K^2+2ay_K+a^2 = x_K^2-2x_Kx_E+x_E^2+y_K^2-2y_Ky_E+y_E^2$.
$2ay_K+a^2 = -2x_Kx_E+x_E^2-2y_Ky_E+y_E^2$.
Since $E$ is on $\Omega$, $x_E^2+y_E^2=2mx_E+a^2$.
$2ay_K+a^2 = -2x_Kx_E+2mx_E+a^2-2y_Ky_E$.
$2ay_K = -2x_Kx_E+2mx_E-2y_Ky_E$.
$ay_K = -x_Kx_E+mx_E-y_Ky_E$.
$ay_K = -x_Kx_E+mx_E-y_K(k_{AP}x_E+a) \implies 2ay_K = -x_Kx_E+mx_E-y_Kk_{AP}x_E$.
$x_E(x_K+y_Kk_{AP}) + 2ay_K = mx_E$. Dividing by $x_E \neq 0$:
(1) $x_K + y_K(k_{AP}+\frac{2a}{x_E}) = m$.
A similar derivation for the perpendicular bisector of $BF$ (using point $F$ on $\Gamma$) yields:
(2) $x_K + y_K(k_{AP}+\frac{2a}{x_F}) = n$.
Subtracting (1) from (2): $y_K(\frac{2a}{x_F}-\frac{2a}{x_E}) = n-m \implies 2ay_K\frac{x_E-x_F}{x_Ex_F} = n-m$.
Using $x_E-x_F = \frac{2(m-n)}{1+k_{AP}^2}$, we get $2ay_K \frac{2(m-n)/(1+k_{AP}^2)}{x_Ex_F} = n-m$. Since $m \neq n$, we find $y_K = -\frac{x_Ex_F(1+k_{AP}^2)}{4a} = -\frac{(m-ak_{AP})(n-ak_{AP})}{a(1+k_{AP}^2)}$.
Using Identities 1, 2, 3:
$y_K = -\frac{1}{a} \left(\frac{(c^2+a^2)(c-d)}{2c(c+d)}\right) \left(\frac{(d^2+a^2)(d-c)}{2d(c+d)}\right) \left/ \frac{(c^2+a^2)(d^2+a^2)}{a^2(c+d)^2} \right. = \frac{a(c-d)^2}{4cd}$.
From (1) and Identity 4: $x_K = m - y_K(k_{AP}+\frac{2a}{x_E}) = m - y_K\frac{cd+a^2}{a(c-d)}$.
$x_K = \frac{c^2-a^2}{2c} - \frac{a(c-d)^2}{4cd} \frac{cd+a^2}{a(c-d)} = \frac{c^2-a^2}{2c} - \frac{(c-d)(cd+a^2)}{4cd} = \frac{2d(c^2-a^2) - (c-d)(cd+a^2)}{4cd} = \frac{cd(c+d)-a^2(c+d)}{4cd} = \frac{(cd-a^2)(c+d)}{4cd}$.
The radius squared of the circumcircle of $\triangle BEF$ is $R_K^2 = KB^2 = x_K^2+(y_K+a)^2$.
$y_K+a = \frac{a(c-d)^2}{4cd}+a = \frac{a(c+d)^2}{4cd}$.
$R_K^2 = \left(\frac{(cd-a^2)(c+d)}{4cd}\right)^2 + \left(\frac{a(c+d)^2}{4cd}\right)^2 = \frac{(c+d)^2}{(4cd)^2} \left((cd-a^2)^2 + a^2(c+d)^2\right) = \frac{(c+d)^2(c^2+a^2)(d^2+a^2)}{16c^2d^2}$.

**7. The Tangency Proof**

The line $\ell_H$ passes through $H(x_H, y_H)$ and is parallel to $AP$. Its equation is $y-y_H = k_{AP}(x-x_H)$, which can be written as $k_{AP}x - y - (k_{AP}x_H - y_H) = 0$.
This line is tangent to the circumcircle of $\triangle BEF$ (center $K$, radius $R_K$) if the square of the distance from $K$ to $\ell_H$ is $R_K^2$. The squared distance is $\frac{(k_{AP}x_K - y_K - (k_{AP}x_H - y_H))^2}{k_{AP}^2+1}$.
The condition for tangency is $(k_{AP}(x_K-x_H)-(y_K-y_H))^2 = R_K^2(1+k_{AP}^2)$.
Let's compute the terms on the left side (LHS).
$x_K-x_H = \frac{(cd-a^2)(c+d)}{4cd} - \frac{c+d}{2} = \frac{(c+d)(cd-a^2-2cd)}{4cd} = -\frac{(c+d)(cd+a^2)}{4cd}$.
$y_K-y_H = \frac{a(c-d)^2}{4cd} - \left(-\frac{a(a^2+cd)}{2cd}\right) = \frac{a(c^2-2cd+d^2)+2a(a^2+cd)}{4cd} = \frac{a(c^2+d^2+2a^2)}{4cd}$.
$k_{AP}(x_K-x_H) = \frac{cd-a^2}{a(c+d)} \left(-\frac{(c+d)(cd+a^2)}{4cd}\right) = -\frac{(cd-a^2)(cd+a^2)}{4acd} = -\frac{c^2d^2-a^4}{4acd}$.
$k_{AP}(x_K-x_H)-(y_K-y_H) = -\frac{c^2d^2-a^4}{4acd} - \frac{a^2(c^2+d^2+2a^2)}{4acd} = -\frac{c^2d^2-a^4+a^2c^2+a^2d^2+2a^4}{4acd} = -\frac{(c^2+a^2)(d^2+a^2)}{4acd}$.
LHS = $\left(-\frac{(c^2+a^2)(d^2+a^2)}{4acd}\right)^2 = \frac{(c^2+a^2)^2(d^2+a^2)^2}{16a^2c^2d^2}$.
Now we compute the right side (RHS) using $R_K^2$ and Identity 3:
RHS = $R_K^2(1+k_{AP}^2) = \left(\frac{(c+d)^2(c^2+a^2)(d^2+a^2)}{16c^2d^2}\right) \left(\frac{(c^2+a^2)(d^2+a^2)}{a^2(c+d)^2}\right) = \frac{(c^2+a^2)^2(d^2+a^2)^2}{16a^2c^2d^2}$.
Since LHS = RHS, the tangency condition is satisfied. This completes the proof.

\subsection{Problem 3}

\begin{problem}
Let $\mathbb{N}$ denote the set of positive integers. A function $f : \mathbb{N} \to \mathbb{N}$ is said to be \textit{bonza} if
\[
f(a) \mid b^a - f(b)^{f(a)}
\]
for all positive integers $a$ and $b$.

Determine the smallest real constant $c$ such that $f(n) \le cn$ for all bonza functions $f$ and all positive integers $n$.
\end{problem}

\begin{proof} [Solution without hint]
Let $P(a, b)$ be the assertion $f(a) \mid b^a - f(b)^{f(a)}$.

**Step 1: Prove that $f(1)=1$**
Let $C=f(1)$. Assume for contradiction that $C>1$. Let $p$ be a prime divisor of $C$.
The assertion $P(p, 1)$ states $f(p) \mid 1^p - f(1)^{f(p)}$, which is $f(p) \mid 1 - C^{f(p)}$.
Since $p \mid C$, we have $C \equiv 0 \pmod p$, so $1 - C^{f(p)} \equiv 1 \pmod p$.
If $p$ were a divisor of $f(p)$, then $p$ would have to divide $1 - C^{f(p)}$, which is impossible as $1 - C^{f(p)} \equiv 1 \pmod p$. Thus, $p \nmid f(p)$.

Now consider the assertion $P(1, p)$, which states $f(1) \mid p - f(p)^{f(1)}$, i.e., $C \mid p - f(p)^C$.
Since $p \mid C$, this implies $p \mid p - f(p)^C$, which in turn implies $p \mid f(p)^C$.
As $p$ is prime, this means $p \mid f(p)$.
We have derived a contradiction: $p \nmid f(p)$ and $p \mid f(p)$.
Therefore, the assumption $C>1$ must be false. Since $f(1) \in \mathbb N$, we must have $f(1)=1$.

**Step 2: Properties of bonza functions**
Let $S_f = \{p \text{ prime} \mid \exists n \in \mathbb N, p \mid f(n)\}$.

**Property 1:** For any $p \in S_f$ and $b \in \mathbb{N}$, if $p \mid f(b)$, then $p \mid b$.
*Proof:* Let $p \in S_f$. By definition, there exists an $a_0 \in \mathbb{N}$ such that $p \mid f(a_0)$. The bonza condition $P(a_0, b)$ is $f(a_0) \mid b^{a_0} - f(b)^{f(a_0)}$. As $p \mid f(a_0)$, we have $p \mid b^{a_0} - f(b)^{f(a_0)}$. If we assume $p \mid f(b)$, this implies $p \mid b^{a_0}$, and since $p$ is prime, $p \mid b$.

**Lemma A:** For any prime $p \in S_f$, we have $f(b) \equiv b \pmod p$ for all $b \in \mathbb N$.
*Proof:* Let $p \in S_f$. First, we show $f(p)$ is a power of $p$. If $f(p)=1$, then for any $a$ with $p \mid f(a)$, $P(a,p) \implies f(a) \mid p^a-f(p)^{f(a)} = p^a-1$, which contradicts $p \mid f(a)$. So $f(p)>1$. Let $q$ be a prime divisor of $f(p)$. Then $q \in S_f$. By Property 1, since $q \mid f(p)$, we must have $q \mid p$. As $p,q$ are primes, $q=p$. Thus $f(p)=p^k$ for some $k \ge 1$.
Now, $P(p,b) \implies f(p) \mid b^p - f(b)^{f(p)}$, so $p^k \mid b^p - f(b)^{p^k}$. This implies $b^p \equiv f(b)^{p^k} \pmod p$. By Fermat's Little Theorem, $b^p \equiv b \pmod p$ and $f(b)^{p^k} \equiv f(b) \pmod p$. Thus, $b \equiv f(b) \pmod p$.

**Lemma B:** For any prime $q \notin S_f$, we have $f(q)=1$.
*Proof:* Let $q$ be a prime with $q \notin S_f$. Suppose $f(q)>1$. Let $p$ be a prime divisor of $f(q)$. Then $p \in S_f$. By Property 1, since $p \mid f(q)$, we must have $p \mid q$. As $p,q$ are primes, $p=q$. This contradicts the fact that $p \in S_f$ and $q \notin S_f$. Thus, $f(q)=1$.

**Step 3: Classification of bonza functions**
For any $p \in S_f$ and any prime $q \notin S_f$, by Lemma A, $f(q) \equiv q \pmod p$. By Lemma B, $f(q)=1$. Thus, we have the crucial condition:
$$ \forall p \in S_f, \forall q \notin S_f \text{ (prime)}, \quad q \equiv 1 \pmod p \quad (*) $$
We analyze the possible structures for the set of primes $S_f$.
1.  $S_f = \emptyset$: Condition (*) is vacuously true. This is a possible case.
2.  $S_f$ is the set of all primes: Condition (*) is vacuously true as there are no primes $q \notin S_f$. This is a possible case.
3.  $S_f$ is a non-empty proper subset of the set of all primes.
    a) Suppose $S_f$ is an infinite proper subset. Let $q$ be a prime such that $q \notin S_f$. By (*), $q \equiv 1 \pmod p$ for all $p \in S_f$. This means every prime $p \in S_f$ is a divisor of $q-1$. Since $S_f$ is infinite, this implies the integer $q-1$ has infinitely many distinct prime divisors, which is impossible. Thus, $S_f$ cannot be an infinite proper subset.
    b) Suppose $S_f$ is a finite, non-empty proper subset. Let $S_f = \{p_1, \dots, p_r\}$. Let $P = p_1 p_2 \dots p_r$. Condition (*) implies that any prime $q \notin S_f$ must satisfy $q \equiv 1 \pmod P$.
    If $P > 2$, then Euler's totient function $\phi(P) > 1$. This means there exists an integer $a$ with $1 < a < P$ and $\gcd(a, P)=1$. By Dirichlet's theorem on arithmetic progressions, there are infinitely many primes $q$ of the form $kP+a$. We can choose such a prime $q$ that is not in the finite set $S_f$. For this prime $q$, we have $q \notin S_f$, so by our condition, it must satisfy $q \equiv 1 \pmod P$. But we chose $q$ such that $q \equiv a \pmod P$. This leads to $a \equiv 1 \pmod P$, which contradicts our choice of $a$.
    Therefore, the assumption $P>2$ must be false. We must have $P \le 2$.
    Since $S_f$ is non-empty, $P \ge 2$. Thus, $P=2$, which implies $S_f=\{2\}$.
Thus, the only possibilities for $S_f$ are $\emptyset$, $\{2\}$, or the set of all primes.

**Step 4: Analysis of $f(n)/n$ for each case**
*   **Case 1: $S_f = \emptyset$.** This implies that for all $n$, $f(n)$ has no prime factors, so $f(n)=1$. The function $f(n)=1$ for all $n$ is bonza, since $1 \mid b^a - 1^{1}$ is always true. For this function, $f(n)/n = 1/n \le 1$.
*   **Case 2: $S_f$ is the set of all primes.** For any prime $p$, Lemma A gives $f(n) \equiv n \pmod p$. This means $p \mid (f(n)-n)$ for all primes $p$. If $f(n) \ne n$, then $f(n)-n$ is a non-zero integer. A non-zero integer can only have a finite number of prime divisors. This forces $f(n)-n=0$, so $f(n)=n$. The function $f(n)=n$ is bonza, since $a \mid b^a - b^a$ is true. For this function, $f(n)/n = 1$.
*   **Case 3: $S_f = \{2\}$.** The range of $f$ is a subset of $\{1\} \cup \{2^k \mid k \in \mathbb N\}$. By Lemma A, $f(n) \equiv n \pmod 2$. This implies $f(n)=1$ for odd $n$, and $f(n)=2^{k_n}$ for some $k_n \ge 1$ for even $n$.
    Let $n$ be an even integer. Let $v_2(m)$ be the exponent of 2 in the prime factorization of $m$.
    1.  For any odd $b$, $P(n,b) \implies f(n) \mid b^n-f(b)^{f(n)}$. Since $b$ is odd, $f(b)=1$. So $2^{k_n} \mid b^n-1$. This must hold for all odd $b$. Thus, $k_n \le \min_{b \text{ odd}} v_2(b^n-1)$.
        Let $n=2^s t$ with $t$ odd, $s \ge 1$. For any odd $b$, let $x=b^t$, which is also odd. We have $v_2(b^n-1) = v_2(x^{2^s}-1)$. It is a known property that for an odd integer $x$ and $k \ge 1$, $v_2(x^{2^k}-1) = v_2(x^2-1)+k-1$. Applying this with $k=s$, we get $v_2(b^n-1) = v_2((b^t)^2-1)+s-1$.
        To find the minimum value, we must minimize $v_2((b^t)^2-1)$ over odd $b$. The map $\phi_t: (\mathbb Z/8\mathbb Z)^\times \to (\mathbb Z/8\mathbb Z)^\times$ given by $\phi_t(y)=y^t$ is a permutation for any odd $t$. This is because for any $y \in (\mathbb Z/8\mathbb Z)^\times$, $y^2 \equiv 1 \pmod 8$, so for $t=2k+1$, $y^t = y^{2k+1} = (y^2)^k y \equiv y \pmod 8$. Thus, as $b$ runs through odd integers, $b^t \pmod 8$ also takes on all values in $\{1,3,5,7\}$. We can choose $b$ such that $b^t \equiv 3 \pmod 8$. For such $b$, $v_2((b^t)^2-1) = v_2(3^2-1)=v_2(8)=3$. The minimum value of $v_2((b^t)^2-1)$ for $b^t \not\equiv 1 \pmod 8$ is 3.
        So, $\min_{b \text{ odd}} v_2(b^n-1) = 3+s-1=s+2=v_2(n)+2$. Thus, $k_n \le v_2(n)+2$.
    2.  For $b=2$, $P(n,2) \implies f(n) \mid 2^n - f(2)^{f(n)}$. Let $f(2)=2^{k_2}$ for some integer $k_2 \ge 1$. The condition is $2^{k_n} \mid 2^n - 2^{k_2 2^{k_n}}$. This requires $k_n \le v_2(2^n - 2^{k_2 2^{k_n}})$.
        If $n \ne k_2 2^{k_n}$, then $v_2(2^n - 2^{k_2 2^{k_n}}) = \min(n, k_2 2^{k_n})$. The condition becomes $k_n \le \min(n, k_2 2^{k_n})$, which implies $k_n \le n$.
        If $n = k_2 2^{k_n}$, the divisibility is on $0$, which holds for any $k_n$. However, $n = k_2 2^{k_n}$ with $k_2 \ge 1$ implies $n \ge 2^{k_n}$. Since $2^x > x$ for all $x \ge 1$, we have $n > k_n$.
        In all cases, we must have $k_n \le n$.
    Combining these constraints, for any bonza function in this class, $f(n) \le 2^{\min(n, v_2(n)+2)}$ for even $n$.

**Step 5: Construction of the maximal function and determination of $c$**
Let's define a function $f_0$ based on the derived upper bound:
$$ f_0(n) = \begin{cases} 1 & \text{if } n \text{ is odd} \\ 2^{\min(n, v_2(n)+2)} & \text{if } n \text{ is even} \end{cases} $$
Let $k_m = \min(m, v_2(m)+2)$ for any even $m$. We verify that $f_0$ is a bonza function.
-   If $a$ is odd, $f_0(a)=1$. The condition is $1 \mid b^a - f_0(b)^{1}$, which is always true.
-   If $a$ is even and $b$ is odd: $f_0(a)=2^{k_a}, f_0(b)=1$. We need $2^{k_a} \mid b^a-1$. By definition, $k_a \le v_2(a)+2$. As established in Step 4, for any even $a$ and odd $b$, $v_2(b^a-1) \ge v_2(a)+2$. Thus, $k_a \le v_2(a)+2 \le v_2(b^a-1)$, so the condition holds.
-   If $a, b$ are even: $f_0(a)=2^{k_a}, f_0(b)=2^{k_b}$. We need $2^{k_a} \mid b^a - (2^{k_b})^{2^{k_a}}$. Let $E = b^a - 2^{k_b 2^{k_a}}$. We need to show $v_2(E) \ge k_a$.
    Let $v_2(b^a) = a v_2(b)$ and the exponent of 2 in the second term is $k_b 2^{k_a}$.
    If $v_2(b^a) \ne k_b 2^{k_a}$, then $v_2(E) = \min(v_2(b^a), k_b 2^{k_a})$. We have $k_a \le a \le a v_2(b) = v_2(b^a)$ since $v_2(b) \ge 1$. Also $k_b \ge 1$, so $k_b 2^{k_a} \ge 2^{k_a} \ge k_a$ for $k_a \ge 1$. Thus $k_a \le v_2(E)$.
    If $v_2(b^a) = k_b 2^{k_a}$, let $v_b=v_2(b)$ and $m=b/2^{v_b}$. The condition is $a v_b = k_b 2^{k_a}$.
    $E = b^a - 2^{k_b 2^{k_a}} = (m 2^{v_b})^a - 2^{a v_b} = m^a 2^{a v_b} - 2^{a v_b} = 2^{a v_b}(m^a-1)$.
    If $m=1$, $b$ is a power of 2, then $E=0$ and $2^{k_a} \mid 0$ holds.
    If $m>1$, $v_2(E) = a v_b + v_2(m^a-1)$. Since $k_a \le a \le a v_b$, and $v_2(m^a-1) \ge 1$ (as $a$ is even, $m$ is odd), we have $k_a \le v_2(E)$.
    Thus, $f_0$ is a bonza function.

Now we find $\sup_{n} f_0(n)/n$. For odd $n$, $f_0(n)/n = 1/n \le 1$. For even $n=2^s t$ with $t$ odd, $s \ge 1$:
$\frac{f_0(n)}{n} = \frac{2^{\min(2^s t, s+2)}}{2^s t}$.
-   If $t \ge 3$, then $2^s t > s+2$ for $s \ge 1$. (Proof: $g(s)=3 \cdot 2^s - s - 2$ is increasing for $s \ge 1$ and $g(1)=3>0$. So $3 \cdot 2^s > s+2$. Then $t \cdot 2^s \ge 3 \cdot 2^s > s+2$.)
    So $\frac{f_0(n)}{n} = \frac{2^{s+2}}{2^s t} = \frac{4}{t} \le \frac{4}{3}$.
-   If $t=1$, $n=2^s$. $\frac{f_0(2^s)}{2^s} = \frac{2^{\min(2^s, s+2)}}{2^s}$.
    The inequality $2^s \ge s+2$ holds for all integers $s \ge 2$. (Proof by induction: for $s=2$, $4 \ge 4$. Assume $2^k \ge k+2$ for $k \ge 2$, then $2^{k+1}=2 \cdot 2^k \ge 2(k+2)=2k+4 = (k+3)+(k+1) > k+3$.)
    For $s=1$, $n=2$: $\frac{f_0(2)}{2} = \frac{2^{\min(2,3)}}{2} = \frac{2^2}{2} = 2$.
    For $s \ge 2$, since $2^s \ge s+2$, we have $\min(2^s, s+2) = s+2$. So $\frac{f_0(2^s)}{2^s} = \frac{2^{s+2}}{2^s} = 4$.
This value of 4 is achieved for $n=4, 8, 16, \dots$. For example, $f_0(4)=2^{\min(4, v_2(4)+2)} = 2^{\min(4,4)} = 2^4=16$, so $f_0(4)/4=4$.

The supremum of $f(n)/n$ over all bonza functions is 4. For any constant $c<4$, the function $f_0$ and $n=4$ gives $f_0(4)=16 > 4c$, so $f_0(4)>cn$. Thus, the smallest constant is $c=4$.
\end{proof}

\subsection{Problem 4}

\begin{problem}
A proper divisor of a positive integer $N$ is a positive divisor of $N$ other than $N$ itself.

The infinite sequence $a_1, a_2, \ldots$ consists of positive integers, each of which has at least three proper divisors. For each $n \ge 1$, the integer $a_{n+1}$ is the sum of the three largest proper divisors of $a_n$.

Determine all possible values of $a_1$.
\end{problem}

\begin{proof} [Solution without hint]
Let $f(N)$ be the sum of the three largest proper divisors of a positive integer $N$. The sequence is defined by $a_{n+1} = f(a_n)$ for $n \ge 1$. The problem requires that each $a_n$ has at least three proper divisors. A proper divisor of $N$ is any positive divisor of $N$ except $N$ itself. The condition that $N$ has at least three proper divisors means that the number of divisors of $N$, denoted by $d(N)$, must satisfy $d(N)-1 \ge 3$, so $d(N) \ge 4$.

Let the divisors of $N$ in increasing order be $1=d_1 < d_2 < d_3 < \dots < d_k=N$, where $k=d(N) \ge 4$. The proper divisors of $N$ are $\{d_1, d_2, \dots, d_{k-1}\}$. The three largest proper divisors are $d_{k-1}, d_{k-2}, d_{k-3}$.
The map $d \mapsto N/d$ is a bijection on the set of divisors of $N$ that reverses the order. Therefore, the largest proper divisor is $d_{k-1} = N/d_2$, the second largest is $d_{k-2} = N/d_3$, and the third largest is $d_{k-3} = N/d_4$. This holds for any $N$ with $d(N) \ge 4$.
Thus, the sum of the three largest proper divisors is given by:
$f(N) = d_{k-1} + d_{k-2} + d_{k-3} = \frac{N}{d_2} + \frac{N}{d_3} + \frac{N}{d_4} = N\left(\frac{1}{d_2}+\frac{1}{d_3}+\frac{1}{d_4}\right)$.

**Lemma 1:** For any valid sequence, $a_n$ must be even for all $n \ge 1$.
*   **Proof:** Suppose $a_k$ is an odd integer for some $k \ge 1$. All its divisors are odd, so its three largest proper divisors are odd. Their sum, $a_{k+1} = f(a_k)$, is also odd. Therefore, if any term is odd, all subsequent terms are odd.
    Let $N$ be an odd integer with $d(N) \ge 4$. Its smallest divisors greater than 1 are $d_2, d_3, d_4$. Since $N$ is odd, all its divisors are odd. The smallest possible value for $d_2$ is 3. Thus, $d_2 \ge 3$. Since $d_3 > d_2$, $d_3 \ge 5$. Similarly, $d_4 > d_3$, so $d_4 \ge 7$.
    The sum $\frac{1}{d_2}+\frac{1}{d_3}+\frac{1}{d_4}$ is maximized when $d_2, d_3, d_4$ are minimized. Thus, $\frac{1}{d_2}+\frac{1}{d_3}+\frac{1}{d_4} \le \frac{1}{3}+\frac{1}{5}+\frac{1}{7} = \frac{71}{105}$.
    Therefore, $f(N) \le \frac{71}{105}N < N$.
    So, if $a_k$ is odd, the sequence $(a_n)_{n \ge k}$ is a strictly decreasing sequence of positive integers. Such a sequence must terminate, meaning it must produce a term which does not satisfy $d(a_n) \ge 4$. Therefore, an infinite sequence is not possible if any term is odd.

**Lemma 2:** For any valid sequence, $a_n$ must be divisible by 3 for all $n \ge 1$.
*   **Proof:** Suppose there is a term $a_k$ not divisible by 3. By Lemma 1, $a_k$ is even.
    Let $a_m$ be any term with $v_3(a_m)=0$. Its smallest divisor is $d_2=2$. Its divisors $d_3, d_4$ are not multiples of 3. Thus $d_3 \ge 4$ and $d_4 \ge 5$.
    $a_{m+1} = a_m(\frac{1}{2}+\frac{1}{d_3}+\frac{1}{d_4}) \le a_m(\frac{1}{2}+\frac{1}{4}+\frac{1}{5}) = \frac{19}{20}a_m < a_m$.
    So, if a term is not divisible by 3, the next term is strictly smaller.

    Now, consider the sequence $(a_n)_{n \ge k}$.
    If for all $n \ge k$, $v_3(a_n)=0$, then the sequence is strictly decreasing. A strictly decreasing sequence of positive integers must terminate. This is a failure.
    Therefore, for the sequence to be infinite, there must be a first term $a_m$ (with $m \ge k$) such that $v_3(a_m)=0$ and $v_3(a_{m+1})>0$.
    For $a_{m+1}$ to gain a factor of 3, the numerator of the fraction in $a_{m+1} = a_m(\frac{1}{d_2}+\frac{1}{d_3}+\frac{1}{d_4})$ must be divisible by 3. With $d_2=2$, this means $d_3d_4+2d_4+2d_3 \equiv 0 \pmod 3$, which implies $(d_3-1)(d_4-1) \equiv 1 \pmod 3$. This holds if and only if $d_3 \equiv 2 \pmod 3$ and $d_4 \equiv 2 \pmod 3$.
    If $v_2(a_m) \ge 2$, then $d_3=4 \equiv 1 \pmod 3$. The condition is not met.
    So, for the transition to happen, the term $a_m$ must have $v_2(a_m)=1$.
    In this case, $d_3=p$, the smallest odd prime factor of $a_m$. We need $p \equiv 2 \pmod 3$.
    $d_4$ is the smallest divisor of $a_m$ greater than $p$. We also need $d_4 \equiv 2 \pmod 3$.
    Let the prime factorization of $a_m$ be $2 \cdot p^{e_p} \cdot q^{e_q} \cdots$, where $p < q < \dots$ are odd primes.
    The candidates for $d_4$ are the smallest divisors of $a_m$ greater than $p$. These are $p^2$ (if $e_p \ge 2$), $2p$, and $q$ (if $a_m$ has a second odd prime factor $q$). Any other divisor of $a_m$ greater than $p$ is larger than one of these three.
    We check the congruences modulo 3, given $p \equiv 2 \pmod 3$:
    - $p^2 \equiv 2^2 = 4 \equiv 1 \pmod 3$.
    - $2p \equiv 2(2) = 4 \equiv 1 \pmod 3$.
    So neither $p^2$ nor $2p$ can be $d_4$.
    This implies that $d_4$ must be $q$, the second smallest odd prime factor of $a_m$. For this to be the case, we must have $q < p^2$ and $q < 2p$. And for the condition to be met, we must have $q \equiv 2 \pmod 3$.
    So, if a term $a_m$ with $v_2=1, v_3=0$ gains a factor of 3, its smallest two odd prime factors, $p$ and $q$, must both be congruent to $2 \pmod 3$.
    In this case, $d_3=p$ and $d_4=q$. Both are odd primes.
    Let's check $v_2(a_{m+1}) = v_2(a_m) + v_2(pq+2(p+q)) - v_2(2pq)$.
    We have $v_2(a_m)=1$. Since $p, q$ are odd, $pq$ is odd. $p+q$ is a sum of two odd numbers, so it's even. $2(p+q)$ is a multiple of 4. So $pq+2(p+q)$ is odd + (multiple of 4), which is odd. Thus $v_2(pq+2(p+q)) = 0$. Also, $v_2(2pq) = 1$ since $p,q$ are odd.
    $v_2(a_{m+1}) = 1 + 0 - 1 = 0$.
    So $a_{m+1}$ is odd. By Lemma 1, this leads to failure.
    In summary, if any term is not divisible by 3, the sequence must fail.

**Lemma 3:** For any valid sequence, no term $a_n$ can be divisible by 5.
*   **Proof:** By Lemmas 1 and 2, any term $a_n$ must be divisible by 2 and 3.
    First, suppose $v_5(a_k)=0$ for some $k \ge 1$. We show that $v_5(a_{k+1})=0$.
    The smallest divisors of $a_k$ are $d_2=2, d_3=3$. $d_4$ is the smallest divisor of $a_k$ greater than 3. Let $p$ be the smallest prime factor of $a_k$ other than 2 or 3. Since $v_5(a_k)=0$, $p \ge 7$. The candidates for $d_4$ are divisors of $a_k$ smaller than $p$, which can only be composed of primes 2 and 3. The smallest such divisor greater than 3 is 4.
    - If $v_2(a_k) \ge 2$, then 4 is a divisor of $a_k$. Since $3 < 4 < p$, we have $d_4=4$.
    - If $v_2(a_k) = 1$, then 4 is not a divisor of $a_k$. The smallest divisor greater than 3 must be $2 \cdot 3 = 6$ or $3^2=9$. Since $6 < 9$ and $6 < p$, $d_4=6$.
    In either case, $d_4$ is not a multiple of 5.
    If $d_4=4$, $a_{k+1} = \frac{13}{12}a_k$. If $d_4=6$, $a_{k+1}=a_k$. Neither operation introduces a factor of 5. So if $v_5(a_k)=0$, then $v_5(a_{k+1})=0$.

    Now, assume for contradiction that some term is divisible by 5. Let $a_n$ be such a term.
    If $v_2(a_n)=1$, then $d_2=2, d_3=3$. Since $v_5(a_n)\ge 1$, 5 is a divisor. $d_4=\min(6,5)=5$.
    Then $a_{n+1} = a_n\left(\frac{1}{2}+\frac{1}{3}+\frac{1}{5}\right) = \frac{31}{30}a_n$.
    $v_2(a_{n+1}) = v_2(a_n) - v_2(30) = 1-1=0$. So $a_{n+1}$ is odd, which fails by Lemma 1.
    Therefore, for a sequence with a term divisible by 5 to be valid, every term $a_n$ must satisfy $v_2(a_n) \ge 2$.
    This implies $d_4=4$ for all $n$, so $a_{n+1} = \frac{13}{12}a_n$. This means $v_2(a_{n+1}) = v_2(a_n)-2$ and $v_3(a_{n+1}) = v_3(a_n)-1$. This cannot continue indefinitely. The sequence must eventually produce a term $a_m$ with $v_2(a_m)<2$ or $v_3(a_m)<1$. If $v_2(a_m)=1$, the next term is odd. If $v_2(a_m)=0$, the term is odd. If $v_3(a_m)=0$, the term is not divisible by 3. All these cases lead to failure.

**Main Analysis**
From the lemmas, any term $a_n$ in a valid sequence must be of the form $N=2^a 3^b M$, where $a,b \ge 1$ and all prime factors of $M$ are $\ge 7$. We analyze the sequence based on the value of $a=v_2(N)$.

**Case 1: $v_2(N)=1$**
Let $N = 2^1 3^b M$ with $b \ge 1$ and prime factors of $M$ being at least 7. The smallest divisors of $N$ are $d_1=1, d_2=2, d_3=3$. The next smallest divisor is $d_4=\min(2^2, 2\cdot3, 3^2, p)$, where $p$ is the smallest prime factor of $M$. Since $v_2(N)=1$, $4$ is not a divisor. $d_4=\min(6,9,p)$. As $p \ge 7$, $d_4=6$.
Then $f(N) = N\left(\frac{1}{2}+\frac{1}{3}+\frac{1}{6}\right) = N(1) = N$.
Such numbers are fixed points. For $a_1$ to be one of these values, we must check that $d(a_1) \ge 4$.
$d(a_1) = d(2^1 3^b M) = 2(b+1)d(M)$. Since $b \ge 1$, $b+1 \ge 2$. $d(M) \ge 1$.
So $d(a_1) \ge 2(2)(1) = 4$. The condition is satisfied.
Thus, any integer $N=2^1 3^b M$ with $b \ge 1$ and prime factors of $M$ being at least 7 is a possible value for $a_1$.

**Case 2: $v_2(N) \ge 2$**
Let $N = 2^a 3^b M$ with $a \ge 2, b \ge 1$ and prime factors of $M$ being at least 7. The smallest divisors are $d_1=1, d_2=2, d_3=3, d_4=4$.
Then $f(N) = N\left(\frac{1}{2}+\frac{1}{3}+\frac{1}{4}\right) = \frac{13}{12}N$.
Let $a_1$ be such a number. The sequence starts with $a_{s+1} = \frac{13}{12}a_s$ as long as $v_2(a_s) \ge 2$.
This recurrence implies $v_2(a_{s+1})=v_2(a_s)-2$ and $v_3(a_{s+1})=v_3(a_s)-1$. For the sequence to be infinite, it must transition to a fixed point, which requires an iterate $a_k$ to have $v_2(a_k)=1$.
Let $v_2(a_1)=a$. The sequence of 2-adic valuations is $a, a-2, a-4, \dots$. For this sequence to attain the value 1, $a$ must be odd. If $a$ were even, it would eventually become 0, making the term odd and causing failure. So, $a$ must be an odd integer, $a \ge 3$.
The transition to $v_2=1$ occurs at step $s_0 = (a-1)/2$. The term is $a_{s_0+1} = a_{(a+1)/2}$.
$a_{s+1} = \left(\frac{13}{12}\right)^s a_1 = 2^{a-2s} 3^{b-s} 13^s M$.
For $s=s_0=(a-1)/2$, we get $v_2(a_{s_0+1}) = a-2\frac{a-1}{2}=1$.
For this term to be a fixed point, its 3-adic valuation must be at least 1.
$v_3(a_{s_0+1}) = b-s_0 = b-\frac{a-1}{2}$. We need $b-\frac{a-1}{2} \ge 1 \iff b \ge 1+\frac{a-1}{2} = \frac{a+1}{2}$.
This condition also ensures $v_3(a_s) \ge 1$ for all $s \le s_0$.

Finally, we verify that all terms in such a sequence satisfy $d(a_n) \ge 4$.
For $s \in \{0, 1, \dots, \frac{a-3}{2}\}$, the term is $a_{s+1} = 2^{a-2s} 3^{b-s} 13^s M$.
$v_2(a_{s+1}) = a-2s \ge a-2(\frac{a-3}{2}) = 3$.
$v_3(a_{s+1}) = b-s \ge b-\frac{a-3}{2} \ge \frac{a+1}{2}-\frac{a-3}{2} = 2$.
So $d(a_{s+1}) = (v_2(a_{s+1})+1)(v_3(a_{s+1})+1)d(13^s M) \ge (3+1)(2+1)(1) = 12 \ge 4$.
The term $a_{(a+1)/2}$ is the fixed point $2^1 \cdot 3^{b-(a-1)/2} \cdot 13^{(a-1)/2} M$.
$v_2=1$, $v_3 = b-\frac{a-1}{2} \ge 1$.
$d(a_{(a+1)/2}) = (1+1)(b-\frac{a-1}{2}+1)d(13^{(a-1)/2}M) \ge 2(1+1)(1) = 4$.
All terms satisfy the condition.

**Conclusion**
The possible values for $a_1$ are integers $N=2^a 3^b M$ where $a,b \ge 1$ and prime factors of $M$ are $\ge 7$, satisfying one of:
1.  $a=1, b \ge 1$ (fixed points).
2.  $a$ is an odd integer, $a \ge 3$, and $b \ge \frac{a+1}{2}$ (converging to a fixed point).
\end{proof}

\subsection{Problem 5}

\begin{problem}
Alice and Bazza are playing the \textit{inekoalaty} game, a two-player game whose rules depend on a positive real number $\lambda$ which is known to both players. On the $n$th turn of the game (starting with $n = 1$) the following happens:
\begin{itemize}
    \item If $n$ is odd, Alice chooses a nonnegative real number $x_n$ such that
    \[
    x_1 + x_2 + \cdots + x_n \le \lambda n.
    \]
    \item If $n$ is even, Bazza chooses a nonnegative real number $x_n$ such that 
    \[
    x_1^2 + x_2^2 + \cdots + x_n^2 \le n.
    \]
\end{itemize}

If a player cannot choose a suitable number $x_n$, the game ends and the other player wins. If the game goes forever, neither player wins. All chosen numbers are known to both players.

Determine all values of $\lambda$ for which Alice has a winning strategy and all those for which Bazza has a winning strategy.
\end{problem}

\begin{proof} [Solution without hint]
Let $S_n = \sum_{i=1}^n x_i$ and $Q_n = \sum_{i=1}^n x_i^2$.
Alice plays on odd turns $n$, Bazza on even turns $n$.
Alice's move $x_n$ is valid if $S_{n-1}+x_n \le \lambda n$. If $S_{n-1} > \lambda n$, she loses.
Bazza's move $x_n$ is valid if $Q_{n-1}+x_n^2 \le n$. If $Q_{n-1} > n$, he loses.

**Case 1: $\lambda > \frac{\sqrt{2}}{2}$ (Alice has a winning strategy)**

Alice can devise a plan to win on a predetermined turn $2m-1$.
**Alice's Plan:**
1.  Alice chooses an integer $m$ large enough such that $\lambda > \frac{m\sqrt{2}}{2m-1}$. Such an $m$ exists because the function $g(m) = \frac{m\sqrt{2}}{2m-1}$ is strictly decreasing for $m \ge 1$ and $\lim_{m\to\infty} g(m) = \frac{\sqrt{2}}{2}$.
2.  For her turns $2k-1$ where $k=1, \dots, m-1$, Alice chooses $x_{2k-1}=0$.
3.  On turn $2m-1$, Alice will choose a value $x_{2m-1}$ that makes Bazza's next move impossible.

**Analysis of Alice's Plan:**
Alice wins on turn $2m-1$ if she can choose $x_{2m-1} \ge 0$ such that her move is valid and Bazza's next move is not. This is possible if and only if the interval of winning moves for $x_{2m-1}$, which is $(\sqrt{2m - Q_{2m-2}}, \lambda(2m-1) - S_{2m-2}]$, is non-empty. This requires the condition:
$$S_{2m-2} + \sqrt{2m - Q_{2m-2}} < \lambda(2m-1)$$
Bazza's goal is to prevent this. Given Alice's plan ($x_{2k-1}=0$ for $k<m$), Bazza controls the values $y_k = x_{2k}$ for $k=1, \dots, m-1$. These choices determine $S_{2m-2} = \sum_{k=1}^{m-1} y_k$ and $Q_{2m-2} = \sum_{k=1}^{m-1} y_k^2$. Bazza's best defense is to choose his moves to maximize the function $F = S_{2m-2} + \sqrt{2m - Q_{2m-2}}$.

**Lemma:** The maximum value of $F$ that Bazza can achieve is $m\sqrt{2}$.
**Proof:** Let $S = S_{2m-2}$ and $Q = Q_{2m-2}$. Bazza's moves are constrained by $\sum_{i=1}^j y_i^2 \le 2j$ for $j=1, \dots, m-1$. This implies $Q = \sum_{k=1}^{m-1} y_k^2 \le 2(m-1)$.
By the Cauchy-Schwarz inequality, $S^2 = (\sum_{k=1}^{m-1} y_k)^2 \le (m-1) \sum_{k=1}^{m-1} y_k^2 = (m-1)Q$. Thus, $S \le \sqrt{(m-1)Q}$.
So, $F \le \sqrt{(m-1)Q} + \sqrt{2m-Q}$. Let this upper bound be $h(Q)$. We maximize $h(Q)$ for $Q \in [0, 2(m-1)]$.
The derivative $h'(Q) = \frac{\sqrt{m-1}}{2\sqrt{Q}} - \frac{1}{2\sqrt{2m-Q}}$ is positive for $Q < 2(m-1)$. So $h(Q)$ is strictly increasing on its domain. The maximum is at $Q=2(m-1)$.
The maximum value of $h(Q)$ is $h(2(m-1)) = \sqrt{(m-1)2(m-1)} + \sqrt{2m-2(m-1)} = \sqrt{2}(m-1) + \sqrt{2} = m\sqrt{2}$.
This maximum is achieved when $Q=2(m-1)$ and the Cauchy-Schwarz inequality is an equality, which means all $y_k$ are equal. Let $y_k=c$. Then $Q=(m-1)c^2=2(m-1) \implies c=\sqrt{2}$. The sequence of moves $x_{2k}=\sqrt{2}$ for $k=1, \dots, m-1$ is valid for Bazza and it maximizes the defensive function $F$.

**Alice's Victory:**
Alice's strategy is guaranteed to work if her winning condition holds even against Bazza's best defense. This requires $\max(F) < \lambda(2m-1)$, which is $m\sqrt{2} < \lambda(2m-1)$, or $\lambda > \frac{m\sqrt{2}}{2m-1}$. By her initial choice of $m$, this condition is met. We must also check that Alice's moves $x_{2k-1}=0$ for $k<m$ are valid. This requires $S_{2k-2} \le \lambda(2k-1)$. Bazza's best defense maximizes $S_{2k-2}$ to $(k-1)\sqrt{2}$. The condition is $(k-1)\sqrt{2} \le \lambda(2k-1)$, or $\lambda \ge \frac{(k-1)\sqrt{2}}{2k-1}$. Since $\lambda > \frac{\sqrt{2}}{2}$ and $\frac{(k-1)\sqrt{2}}{2k-1}$ is an increasing function of $k$ with limit $\frac{\sqrt{2}}{2}$, this condition holds for all $k$. Thus, Alice has a winning strategy.

**Case 2: $\lambda < \frac{\sqrt{2}}{2}$ (Bazza has a winning strategy)**

Bazza's strategy is to always play $x_{2k} = \sqrt{2k - Q_{2k-1}}$ if possible. This sets $Q_{2k}=2k$. As shown in Case 1, Alice cannot win against this strategy because her winning condition $\lambda > \frac{m\sqrt{2}}{2m-1}$ can never be met if $\lambda < \frac{\sqrt{2}}{2}$. We now show that Bazza will win.

Alice loses on turn $2m-1$ if $S_{2m-2} > \lambda(2m-1)$. To survive, Alice must choose her moves to keep the sequence of sums $S_{2k-2}$ as small as possible for as long as possible.
With Bazza's strategy, Alice's choice of $x_{2k-1}$ (provided $x_{2k-1} \le \sqrt{2}$) determines Bazza's response $x_{2k}=\sqrt{2-x_{2k-1}^2}$. The sum grows by $C_k = x_{2k-1} + \sqrt{2-x_{2k-1}^2}$ over turns $2k-1$ and $2k$.
To minimize the sum $S_{2m-2} = \sum_{k=1}^{m-1} C_k$, Alice must choose each $x_{2k-1}$ to minimize $C_k$. The function $f(x)=x+\sqrt{2-x^2}$ on $[0, \sqrt{2}]$ has a minimum value of $\sqrt{2}$, achieved only at $x=0$ and $x=\sqrt{2}$. Any other choice would lead to a strictly larger sum $S_{2m-2}$ for all $m>k+1$, making survival strictly harder. Thus, an optimal survival strategy for Alice must consist only of moves $x_{2k-1} \in \{0, \sqrt{2}\}$.

Let's compare these two choices at turn $2k-1$. Suppose Alice has survived so far, with sum $S_{2k-2}$.
1.  If Alice chooses $x_{2k-1}=0$: This move is valid if $S_{2k-2} \le \lambda(2k-1)$. The resulting sum is $S_{2k} = S_{2k-2} + \sqrt{2}$.
2.  If Alice chooses $x_{2k-1}=\sqrt{2}$: This move is valid if $S_{2k-2}+\sqrt{2} \le \lambda(2k-1)$. The resulting sum is $S_{2k} = S_{2k-2} + \sqrt{2}$.

Both choices lead to the same future sums $S_{2j}$ for $j \ge k$, meaning the survival conditions for all subsequent turns are identical regardless of which of the two is chosen. However, the condition to be allowed to make the choice at turn $2k-1$ is strictly easier for $x_{2k-1}=0$. A strategy involving $x_{2k-1}=\sqrt{2}$ is only valid if the corresponding strategy with $x_{2k-1}=0$ is also valid, but the converse is not true. Therefore, the strategy of always choosing $x_{2k-1}=0$ is Alice's best hope for survival. If she cannot survive with this strategy, she cannot survive with any other.

We now analyze this specific line of play: Alice always plays $x_{2k-1}=0$, and Bazza responds with $x_{2k}=\sqrt{2}$.
1.  Let $h(k) = \frac{(k-1)\sqrt{2}}{2k-1}$. Since $h(k)$ is strictly increasing and approaches $\frac{\sqrt{2}}{2}$, and $\lambda < \frac{\sqrt{2}}{2}$, there exists a smallest integer $m \ge 2$ such that $\lambda < h(m)$.
2.  For any $k < m$, we have $\lambda \ge h(k)$. Alice's move $x_{2k-1}=0$ is valid, since $S_{2k-2}=(k-1)\sqrt{2}$ and the condition is $(k-1)\sqrt{2} \le \lambda(2k-1)$, which is equivalent to $\lambda \ge h(k)$.
3.  On turn $2m-1$, Alice has played according to her optimal survival strategy. The sum is $S_{2m-2}=(m-1)\sqrt{2}$. She must choose $x_{2m-1} \ge 0$ such that $(m-1)\sqrt{2} + x_{2m-1} \le \lambda(2m-1)$.
4.  By the choice of $m$, we have $\lambda < \frac{(m-1)\sqrt{2}}{2m-1}$, which is $\lambda(2m-1) < (m-1)\sqrt{2}$.
5.  The condition for Alice's move becomes $(m-1)\sqrt{2} + x_{2m-1} \le \lambda(2m-1) < (m-1)\sqrt{2}$. This implies $x_{2m-1} < 0$, which is impossible.
Alice cannot make a move, so Bazza wins.

**Case 3: $\lambda = \frac{\sqrt{2}}{2}$ (Draw)**

In this case, neither player has a winning strategy. A player has a winning strategy if they can force a win in a finite number of moves against any of the opponent's strategies.

**Alice does not have a winning strategy.**
To prove this, we show that Bazza has a defensive strategy that prevents Alice from ever winning. Let Bazza adopt the strategy of always choosing $x_{2k} = \sqrt{2k - Q_{2k-1}}$ (if possible).
Alice wins if she can play $x_{2m-1}$ on turn $2m-1$ such that $Q_{2m-1} > 2m$. With Bazza's strategy, $Q_{2m-2}=2(m-1)$ (assuming Alice has not won before). Alice's winning condition becomes $2(m-1)+x_{2m-1}^2 > 2m$, which simplifies to $x_{2m-1} > \sqrt{2}$.
To play such a move, her budget must allow it: $S_{2m-2} + x_{2m-1} \le \lambda(2m-1) = \frac{\sqrt{2}}{2}(2m-1)$.
This requires $S_{2m-2} + \sqrt{2} < \frac{\sqrt{2}}{2}(2m-1)$.
Against Bazza's strategy, $S_{2m-2} = \sum_{k=1}^{m-1} (x_{2k-1} + \sqrt{2-x_{2k-1}^2})$. The minimum value of each term is $\sqrt{2}$, so the minimum possible value of $S_{2m-2}$ is $(m-1)\sqrt{2}$.
Thus, for Alice to win, it must be that $(m-1)\sqrt{2} + \sqrt{2} < \frac{\sqrt{2}}{2}(2m-1)$, which simplifies to $m\sqrt{2} < \frac{\sqrt{2}}{2}(2m-1) \iff 2m < 2m-1 \iff 0 < -1$. This is impossible.
So, Alice can never satisfy her winning condition against this strategy from Bazza. Therefore, Alice does not have a winning strategy.

**Bazza does not have a winning strategy.**
To prove this, we show that Alice has a defensive strategy that prevents Bazza from ever winning. Let Alice adopt the strategy of always choosing $x_{2k-1}=0$.
Bazza wins if he can force $S_{2m-2} > \lambda(2m-1) = \frac{\sqrt{2}}{2}(2m-1)$ for some $m$.
With Alice's strategy, $S_{2m-2} = \sum_{k=1}^{m-1} x_{2k}$. Bazza's moves $x_{2k}$ are constrained by $Q_{2k} = \sum_{i=1}^k x_{2i}^2 \le 2k$.
To force a win, Bazza must maximize $S_{2m-2}$. As shown in the lemma in Case 1, the maximum possible value for $S_{2m-2}$ is $(m-1)\sqrt{2}$, achieved when Bazza plays $x_{2k}=\sqrt{2}$ for all $k<m$.
So, the largest sum Bazza can force is $S_{2m-2}=(m-1)\sqrt{2}$. Bazza wins if this sum exceeds Alice's budget limit, i.e., if $(m-1)\sqrt{2} > \frac{\sqrt{2}}{2}(2m-1)$. This simplifies to $2(m-1) > 2m-1 \iff 2m-2 > 2m-1 \iff -2 > -1$. This is impossible.
So, Bazza can never force Alice to lose. Therefore, Bazza does not have a winning strategy.

Since neither player can force a win, neither has a winning strategy. The game can continue indefinitely, for example with Alice always playing $x_{2k-1}=0$ and Bazza always playing $x_{2k}=\sqrt{2}$. This results in a draw.
\end{proof}

\subsection{Problem 6} \label{a:6}

\begin{problem}
Consider a $2025 \times 2025$ grid of unit squares. Matilda wishes to place on the grid some rectangular tiles, possibly of different sizes, such that each side of every tile lies on a grid line and every unit square is covered by at most one tile.

Determine the minimum number of tiles Matilda needs to place so that each row and each column of the grid has exactly one unit square that is not covered by any tile.
\end{problem}

\begin{proof} [Flawed solution without hint]
Let the grid be an $N \times N$ grid of unit squares, where $N=2025$. Let the square in row $i$ and column $j$ be denoted by $(i,j)$ for $1 \le i,j \le N$.

**1. The Structure of Uncovered Squares**

The condition is that each row and each column of the grid has exactly one uncovered square. Let $U$ be the set of uncovered squares.
For each row $i$, there is a unique column $j$ such that $(i,j) \in U$. This allows us to define a function $\sigma: \{1, \dots, N\} \to \{1, \dots, N\}$, where the uncovered square in row $i$ is $(i, \sigma(i))$.
For each column $j$, there is a unique row $i$ such that $(i,j) \in U$. This means that for any $j_0 \in \{1, \dots, N\}$, there is a unique $i_0$ such that $\sigma(i_0) = j_0$. This implies that the function $\sigma$ is a bijection, i.e., a permutation.
So, the set of uncovered squares is $U = \{(i, \sigma(i)) \mid i=1, \dots, N\}$ for some permutation $\sigma$ of $\{1, \dots, N\}$.

**2. Partitioning the Tiled Region**

Let $C$ be the set of all squares covered by tiles. $C$ is the complement of $U$ in the grid. We can partition $C$ into two disjoint sets:
*   $C_L = \{(i,j) \mid j < \sigma(i)\}$, the set of squares in $C$ that are to the left of the uncovered square in their respective rows.
*   $C_R = \{(i,j) \mid j > \sigma(i)\}$, the set of squares in $C$ that are to the right of the uncovered square in their respective rows.

It is clear that $C = C_L \cup C_R$ and $C_L \cap C_R = \emptyset$.

**3. Tiles are Confined to $C_L$ or $C_R$**

Let a rectangular tile $R$ cover the squares $(i,j)$ for $i \in [r_1, r_2]$ and $j \in [c_1, c_2]$, where $[a,b]$ denotes the set of integers $\{a, a+1, \dots, b\}$. For $R$ to be a valid tile, all squares it covers must be in $C$. This means for any $i \in [r_1, r_2]$, the uncovered square $(i, \sigma(i))$ is not in $R$. This implies that for any $i \in [r_1, r_2]$, $\sigma(i) \notin [c_1, c_2]$.

Let $S_{\sigma} = \{\sigma(i) \mid i \in [r_1, r_2]\}$. The condition for $R$ to be a valid tile is $S_{\sigma} \cap [c_1, c_2] = \emptyset$.
Let $m = \min(S_{\sigma})$ and $M = \max(S_{\sigma})$. The condition is equivalent to either $c_2 < m$ or $c_1 > M$.

*   If $c_2 < m$, then for any square $(i,j) \in R$, we have $j \le c_2 < m \le \sigma(i)$ for all $i \in [r_1, r_2]$. Thus $j < \sigma(i)$, which implies $(i,j) \in C_L$. Therefore, the entire tile $R$ is contained in $C_L$.
*   If $c_1 > M$, then for any square $(i,j) \in R$, we have $j \ge c_1 > M \ge \sigma(i)$ for all $i \in [r_1, r_2]$. Thus $j > \sigma(i)$, which implies $(i,j) \in C_R$. Therefore, the entire tile $R$ is contained in $C_R$.

This proves that any tile must be entirely in $C_L$ or entirely in $C_R$. Consequently, the regions $C_L$ and $C_R$ are tiled independently. The total minimum number of tiles, $k$, is the sum of the minimum number of tiles required to tile $C_L$ ($k_L$) and $C_R$ ($k_R$). So, $k = k_L + k_R$.

**4. Minimum Number of Tiles for $C_L$**

Let's determine $k_L$ for an arbitrary permutation $\sigma$.
Consider the set of squares $S_L = \{(i, \sigma(i)-1) \mid i \in \{1, \dots, N\} \text{ such that } \sigma(i) > 1\}$.
The size of this set is $|S_L| = N-1$, since there is exactly one $i_0$ for which $\sigma(i_0)=1$. Each square in $S_L$ is in $C_L$.

Let $s_i = (i, \sigma(i)-1)$ and $s_j = (j, \sigma(j)-1)$ be any two distinct squares in $S_L$. Assume without loss of generality that $i < j$. Any tile that covers both $s_i$ and $s_j$ must contain the smallest rectangle $R_{ij}$ that contains them. The rows of $R_{ij}$ are $[i,j]$ and the columns are $[\min(\sigma(i)-1, \sigma(j)-1), \max(\sigma(i)-1, \sigma(j)-1)]$.

We investigate if $R_{ij}$ contains any uncovered square from $U$.
*   **Case 1: $\sigma(i) < \sigma(j)$.** The column interval of $R_{ij}$ is $[\sigma(i)-1, \sigma(j)-1]$. Consider the uncovered square $u_i = (i, \sigma(i))$. Its row $i$ is in $[i,j]$. Its column $\sigma(i)$ is in $[\sigma(i)-1, \sigma(j)-1]$ because $\sigma(i)-1 \le \sigma(i)$ and $\sigma(i) < \sigma(j)$ implies $\sigma(i) \le \sigma(j)-1$ for integers. So, $u_i \in R_{ij}$.
*   **Case 2: $\sigma(i) > \sigma(j)$.** The column interval of $R_{ij}$ is $[\sigma(j)-1, \sigma(i)-1]$. Consider the uncovered square $u_j = (j, \sigma(j))$. Its row $j$ is in $[i,j]$. Its column $\sigma(j)$ is in $[\sigma(j)-1, \sigma(i)-1]$ because $\\sigma(j)-1 \le \sigma(j)$ and $\sigma(j) < \sigma(i)$ implies $\sigma(j) \le \sigma(i)-1$. So, $u_j \in R_{ij}$.

In both cases, the rectangle $R_{ij}$ contains an uncovered square. Therefore, $s_i$ and $s_j$ cannot be in the same tile. This holds for any pair of distinct squares in $S_L$. Thus, all $N-1$ squares in $S_L$ must belong to different tiles. This provides a lower bound on the number of tiles for $C_L$: $k_L \ge N-1$.

To show this bound is achievable, we can tile $C_L$ with exactly $N-1$ tiles. For each $i$ such that $\sigma(i)>1$, the region to be tiled in row $i$ is $\{(i,j) \mid 1 \le j \le \sigma(i)-1\}$. This is a single rectangular region, which can be covered by one tile $T_i = [i,i] \times [1, \sigma(i)-1]$. There are $N-1$ such values of $i$, so this construction uses $N-1$ tiles to cover all of $C_L$.

Thus, for any permutation $\sigma$, the minimum number of tiles for $C_L$ is exactly $k_L = N-1$.

**5. Minimum Number of Tiles for $C_R$**

We use a symmetry argument to determine $k_R$. Consider a 180-degree rotation of the grid. A square $(i,j)$ is mapped to $(i',j') = (N+1-i, N+1-j)$. An uncovered square $(i, \sigma(i))$ is mapped to $(N+1-i, N+1-\sigma(i))$.
Let's define a new permutation $\tau$ on $\{1, \dots, N\}$ by $\tau(i') = N+1-\sigma(N+1-i')$. The set of uncovered squares in the rotated grid is $\{(i', \tau(i')) \mid i'=1, \dots, N\}$.
The region $C_R(\sigma) = \{(i,j) \mid j > \sigma(i)\}$ in the original grid corresponds to the region of squares $(i',j')$ where $N+1-j' > \sigma(N+1-i')$. This inequality is equivalent to $j' < N+1-\sigma(N+1-i')$, which is $j' < \tau(i')$. This is precisely the region $C_L(\tau)$ for the permutation $\tau$.
The minimum number of tiles required to tile a region is invariant under rotation. Thus, the minimum number of tiles for $C_R(\sigma)$ is equal to the minimum number of tiles for $C_L(\tau)$.
Since we proved that $k_L(\pi) = N-1$ for *any* permutation $\pi$, it holds for $\tau$. So, $k_R(\sigma) = k_L(\tau) = N-1$.

**6. Conclusion**

For any permutation $\sigma$, the minimum number of tiles is $k = k_L + k_R = (N-1) + (N-1) = 2N-2$.
Since the number of tiles is independent of the choice of permutation, the minimum number of tiles Matilda needs is $2N-2$.

For $N=2025$, the minimum number of tiles is $2 \times (2025-1) = 2 \times 2024 = 4048$.
\end{proof}

\end{document}
